% easychair.tex,v 3.5 2017/03/15

\documentclass{easychair}
%\documentclass[EPiC]{easychair}
%\documentclass[EPiCempty]{easychair}
%\documentclass[debug]{easychair}
%\documentclass[verbose]{easychair}
%\documentclass[notimes]{easychair}
%\documentclass[withtimes]{easychair}
%\documentclass[a4paper]{easychair}
%\documentclass[letterpaper]{easychair}

\usepackage{doc}
\usepackage{minted}
%% \usepackage{amsfonts}
\usepackage{graphicx}
\usepackage{amsmath}
\usepackage{amssymb}
\usepackage{bussproofs}
\usepackage{cite}

% For ≞ (requires the LuaLaTeX engine)
\usepackage{unicode-math}
\setmathfont{Stix Two Math}

% Use this if you have a long article and want to create an index
% \usepackage{makeidx}

% In order to save space or manage large tables or figures in a
% landcape-like text, you can use the rotating and pdflscape
% packages. Uncomment the desired from the below.
%
% \usepackage{rotating}
% \usepackage{pdflscape}

% Some of our commands for this guide.
%
\newcommand{\easychair}{\textsf{easychair}}
\newcommand{\miktex}{MiK{\TeX}}
\newcommand{\texniccenter}{{\TeX}nicCenter}
\newcommand{\makefile}{\texttt{Makefile}}
\newcommand{\latexeditor}{LEd}
\newcommand{\U}{\Theta}
\newcommand{\Space}{\texttt{Space}}
\newcommand{\Term}{\texttt{Term}}
\newcommand{\Type}{\texttt{Type}}
\newcommand{\Bool}{\texttt{Bool}}
\newcommand{\arrow}{\to}
\newcommand{\limp}{\Rightarrow}
\newcommand{\True}{\texttt{True}}
\newcommand{\False}{\texttt{False}}

%\makeindex

%% Front Matter
%%
% Regular title as in the article class.
%
\title{Estimating the Probability of a Proposition to be Provable
  Using Probabilistic Logic Networks}

% Authors are joined by \and. Their affiliations are given by \inst, which indexes
% into the list defined using \institute
%
\author{Nil Geisweiller}

% Institutes for affiliations are also joined by \and,
\institute{
  SingularityNET Foundation,\\
  Zug, Switzerland\\
  \email{nil@singularitynet.io}
}

%  \authorrunning{} has to be set for the shorter version of the authors' names;
% otherwise a warning will be rendered in the running heads. When processed by
% EasyChair, this command is mandatory: a document without \authorrunning
% will be rejected by EasyChair
\authorrunning{Geisweiller}

% \titlerunning{} has to be set to either the main title or its shorter
% version for the running heads. When processed by
% EasyChair, this command is mandatory: a document without \titlerunning
% will be rejected by EasyChair
\titlerunning{Estimating the Probability of a Proposition to be
  Provable Using Probabilistic Logic Networks}

\begin{document}

\maketitle

\begin{abstract}
\end{abstract}

\section{Introduction}

In this paper we will show how to estimate the probability of a
proposition to be provable given available evidence using
Probabilistic Logic Networks (PLN)~\cite{Goertzel09PLN}.  We will then
explain how such estimation can be used as guiding heuristics for
Automated Theorem Proving (ATP).

The idea is to define a ternary predicate holding the relationship
between theory, proof and theorem.  Given its semi-decidable nature,
under finite resources, we cannot hope to establish whether such
relationship hold for any triple of theory, proof and theorem.  We can
however hope to estimate, with various degrees of confidence, the
probability that it may or may not hold given available evidence.  For
instance, a piece of evidence in favor of
$$\forall x \ P(x)$$ could be that $P$ holds for some $a$.
Potentially every piece of information that relates to the conjecture
without contradicting it, should be taken into account to estimate the
probability that it is a theorem.  Although only having an actual
proof, or an actual contradiction, may ever move that probability to a
certainty.  What is more, such ability can then be used as a guide to
discover proofs by prioritizing the search over lemmas that are
themselves more likely to be provable.  The same idea can be applied
on these lemmas till the recursion hopefully bottoms out by reaching
the axioms or a contradiction.

NEXT: add quick state of the art.

\section{Defining the Theory Proof Theorem Predicate}

How can PLN be used to reason about the relationship between Theory,
Proof and Theorem?  Reasoning is treated from a Type Theoretic angle,
thus proofs are terms, theorems are types and theories are sets of
typing relationships between proofs and types.  Let us now define a
ternary predicate $\U$, representing such relationship
$$\U :\ (\Space \times \Term \times \Type) \arrow \Bool$$ where $\Space$
is a collection of typing relationships, $\Term$ is a class of terms
representing proofs and $\Type$ is a class of types representing
theorems.  The content of $\U$ can be specified with PLN statements
expressing rewriting laws within that system.  For instance modus
ponens would be represented as\footnote{Note that due to PLN being in
a state of rework, the syntax used here is provisional.}
$$\U(\Gamma, f, a \to b) \land \U(\Gamma, x, a) \limp \U(\Gamma, f(x),
b)\ \measeq\ \True$$ where $\Gamma$, $f$, $x$, $a$ and $b$ are
variables, universally quantified in this context.  The symbol
$\measeq$ is proper to PLN and can be read as \emph{measured by}.  It
relates a logical PLN statement to a \emph{True Value}.  In the
example above the statement is certain, so the truth value is Boolean.
We will see below that in other situations, true values can be
probabilities, and even second order probabilities (probabilities over
probabilities).  So the full statement can be read as: If $f$ is a
proof of $a \arrow b$, and $x$ is a proof of $a$, then with certainty
$f(x)$ is a proof of $b$ in theory $\Gamma$.  In addition, PLN allows
to reason about uncertain knowledge via Induction and Abduction.
\section{Formulating the Probability of Provability in PLN}
Given a theory $\Gamma$ and a conjecture $C$, the question is: what is
the probability that there exists a proof $p$ of $C$ in $\Gamma$,
which can be formulated by the following PLN query
$$\exists p\ \U(\Gamma, p, C)\ \measeq\ ?\texttt{tv}$$
Thus given
$$\U(\Gamma, pa, P(a))\ \measeq\ \True$$
$$\U(\Gamma, pb, P(b))\ \measeq\ \True$$
$$\vdots$$
one can infer
$$\U(\Gamma, p, \forall x\ P(x))\ \measeq\ \True$$

\section{Probability of Provability as Guiding Heuristic}

\newpage

%------------------------------------------------------------------------------

\bibliographystyle{plain}
\label{sect:bib}
%\bibliographystyle{alpha}
%\bibliographystyle{unsrt}
%\bibliographystyle{abbrv}
\bibliography{ProbEstThrmInfCtrl}

\newpage

%------------------------------------------------------------------------------

\appendix

%------------------------------------------------------------------------------
\end{document}
