% easychair.tex,v 3.5 2017/03/15

\documentclass{easychair}
%\documentclass[EPiC]{easychair}
%\documentclass[EPiCempty]{easychair}
%\documentclass[debug]{easychair}
%\documentclass[verbose]{easychair}
%\documentclass[notimes]{easychair}
%\documentclass[withtimes]{easychair}
%\documentclass[a4paper]{easychair}
%\documentclass[letterpaper]{easychair}

\usepackage{doc}
\usepackage{minted}
%% \usepackage{amsfonts}
\usepackage{graphicx}
\usepackage{amsmath}
\usepackage{amssymb}
\usepackage{bussproofs}
\usepackage{cite}

% For ≞ (requires the LuaLaTeX engine)
\usepackage{unicode-math}
\setmathfont{Stix Two Math}

% Use this if you have a long article and want to create an index
% \usepackage{makeidx}

% In order to save space or manage large tables or figures in a
% landcape-like text, you can use the rotating and pdflscape
% packages. Uncomment the desired from the below.
%
% \usepackage{rotating}
% \usepackage{pdflscape}

% Some of our commands for this guide.
%
\newcommand{\easychair}{\textsf{easychair}}
\newcommand{\miktex}{MiK{\TeX}}
\newcommand{\texniccenter}{{\TeX}nicCenter}
\newcommand{\makefile}{\texttt{Makefile}}
\newcommand{\latexeditor}{LEd}
\newcommand{\Theory}{\texttt{Theory}}
\newcommand{\Proof}{\texttt{Proof}}
\newcommand{\Proposition}{\texttt{Proposition}}
\newcommand{\Bool}{\texttt{Bool}}
\newcommand{\arrow}{\to}
\newcommand{\limp}{\Rightarrow}
\newcommand{\True}{\texttt{True}}
\newcommand{\False}{\texttt{False}}
\newcommand{\STV}[2]{<\!#1, #2\!>}

%\makeindex

%% Front Matter
%%
% Regular title as in the article class.
%
%% \title{Estimating the Probability of a Proposition to be Provable
%%   Using Probabilistic Logic Networks}
%% \title{Estimating the Probability of a Conjecture to be a Theorem with
%%   Probabilistic Logic Networks, and How to Take Advantage of it for
%%   Inference Control}
\title{Estimating the Probability of a Conjecture to be a Theorem with
  PLN for Inference Control}

% Authors are joined by \and. Their affiliations are given by \inst, which indexes
% into the list defined using \institute
%
\author{Nil Geisweiller}

% Institutes for affiliations are also joined by \and,
\institute{
  SingularityNET Foundation,\\
  Zug, Switzerland\\
  \email{nil@singularitynet.io}
}

%  \authorrunning{} has to be set for the shorter version of the authors' names;
% otherwise a warning will be rendered in the running heads. When processed by
% EasyChair, this command is mandatory: a document without \authorrunning
% will be rejected by EasyChair
\authorrunning{Geisweiller}

% \titlerunning{} has to be set to either the main title or its shorter
% version for the running heads. When processed by
% EasyChair, this command is mandatory: a document without \titlerunning
% will be rejected by EasyChair
\titlerunning{Estimating the Probability of a Conjecture to be a Theorem
  with PLN for Inference Control}

\begin{document}

\maketitle

%% \begin{abstract}
%% \end{abstract}

\section{Introduction}

The problem of estimating the probability of a conjecture to be a
theorem has been mentioned in the literature as early as 1954 by
George P\'olya~\cite{Polya1954}.  More formal treatments have appeared
since then such as the work of Scott Garrabrant, Abram Demski et al on
\emph{Uniform Coherence}~\cite{Garrabrant2016, Abram2016} seemingly
the most advanced treatment on the subject to date.  To the best of
our knowledge though, no one has considered using a probabilistic
logic incorporating inductive and abductive reasoning such as
NAL~\cite{Wang2013} or PLN~\cite{Goertzel09PLN}, which we believe is
particularly well suited to this problem.  In this paper we show how
to estimate the probability of a proposition to be a theorem given all
available evidence by using Probabilistic Logic Networks (PLN).  We
then explain how such estimation can be used as guiding heuristics for
Automated Theorem Proving.

The idea we develop here is to define a ternary predicate holding the
relationship between theory, proof and theorem, and to
probabilistically reason about it.  Given its semi-decidable nature we
cannot hope in practice to establish whether such relationship holds
for any argument.  We can however hope to estimate, with various
degrees of confidence, the probability that it may or may not hold
given the available evidence.  For instance, some pieces of evidence
in favor of
$$\forall x \ P(x)$$ could be that $P$ holds for some $a_1, \dots,
a_n$.  Ideally every bit of information that relates to the conjecture
should be taken into account to estimate its probability of being a
theorem.  Although only a rigorous proof, or a contradiction, can
establish certainty.  Moreover, such ability can then be used as a
guide to discover proofs by prioritizing the search over lemmas that
are themselves more likely to be provable.  The same idea can be
applied recursively on these lemmas till it bottoms out by reaching
the axioms, or by exhibiting a contradiction, thereby hopefully
reducing the amount of necessary backtracking.

\section{Relating Theory, Proof and Proposition in PLN}

From a type theoretic perspective, propositions are types, theories
are collections of typing relationships and proofs are terms
inhabiting types.  Let us define a ternary predicate $\Theta$,
representing such relationship
$$\Theta : \Theory \times \Proof \times \Proposition \arrow \Bool$$
where $\Theory$ is a set of collections of typing relationships
encoding the axioms and inference rules of each theory, $\Proof$ is a
set of terms representing proofs, and $\Proposition$ is a set of types
representing propositions.  The content of $\Theta$ can in principle
be characterized in PLN by formalizing the rewriting laws of such type
system.  For instance modus ponens could be formulated
as\footnote{Note that due to PLN being in a state of rework, the
syntax used here is provisional.}
$$\Theta(\Gamma, f, a \to b) \land \Theta(\Gamma, x, a) \limp
\Theta(\Gamma, f(x), b)\ \measeq\ \STV{1}{1}$$ where $\Gamma$, $f$,
$x$, $a$ and $b$ are universally quantified variables, $\to$ is an
implication at the object level, $\measeq$ which can be read as
\emph{measured by} and relates a PLN \emph{statement}, here an
implication at the logical level $\limp$, to a \emph{truth value},
here $\STV{1}{1}$, forming a PLN \emph{judgment} capturing the
uncertainty of the statement.  The first number of the truth value
represents the strength and the second number represents the
confidence over that strength, although underneath, a truth value is a
second order distribution.  In the example above the judgment is
certain because both the strength and the confidence are 1.  The full
judgment can be read as: in theory $\Gamma$, if $f$ is a proof of $a
\arrow b$, and $x$ is a proof of $a$, then with certainty $f(x)$ is a
proof of $b$.  In addition, PLN allows to reason about uncertain
knowledge via induction and abduction.  Given a corpus of examples
(and counter examples) of $\Theta$, induction can be used to gather
statistics about the probability of some propositions meeting some
criteria to be theorems.  Abduction provides a similar mechanism by
considering properties over $\Theta$ instead of examples.

Given a theory $\Gamma$ and a proposition $C$, the question \emph{what
is the probability that there exists a proof $p$ of $C$ in $\Gamma$?}
can be formulated in PLN by the query
$$\exists p\ \Theta(\Gamma, p, C)\ \measeq\ \$\texttt{tv}$$ where
$\$\texttt{tv}$ is a hole corresponding to the truth value to be
filled by PLN.  The way this truth value is calculated may involve
both crisp logical reasoning and statistical reasoning, the latter
including recognizing patterns relating theories, proofs and
propositions.

\section{Estimating Provability as Guiding Heuristic}

The ability to estimate the probability of finding a proof of a
proposition could be used as guiding heuristic.  For instance one may
break up the task of finding a proof of $C$ into two competing paths
each composed of two subtasks:
\begin{itemize}
\item $A$-path: find a proof of $A \to C$, find a proof of $A$,
  then use modus ponens to prove $C$.
\item $B$-path: find a proof of $B \to C$, find a proof of $B$,
  then use modus ponens to prove $C$.
\end{itemize}
To decide whether to take the $A$-path or the $B$-path, one may
formulate the PLN queries
$$\exists p_A\ \Theta(\Gamma, p_A, A) \land \exists
p_{AC}\ \Theta(\Gamma, p_{AC}, A \to C)\ \measeq\ \$\texttt{tv}_A$$
$$\exists p_B\ \Theta(\Gamma, p_B, B) \land \exists
p_{BC}\ \Theta(\Gamma, p_{BC}, B \to C)\ \measeq\ \$\texttt{tv}_B$$ let
PLN reason till both $\$\texttt{tv}_A$ and $\$\texttt{tv}_B$ get
filled with truth values of decent confidences and pick up the path
with the best truth value.  If we want to compare the modus ponens
rule to other inference rules, we can even go further by existentially
quantifying the premise as well, resulting in the PLN query
$$\exists a\ \left(\exists p_a\ \Theta(\Gamma, p_a, a) \land \exists
p_{aC}\ \Theta(\Gamma, p_{aC}, a \to
C)\right)\ \measeq\ \$\texttt{tv}_a$$ If it gets selected the search
will progressively instantiate $a$ into actual premises, breaking up
the query into more specific queries resembling the ones corresponding
to the $A$-path and the $B$-path and so on.  An early prototype of the
ideas described in this paper can be found
in~\cite{Geisweiller2025PICM}.

\newpage

%------------------------------------------------------------------------------

\bibliographystyle{plain}
\label{sect:bib}
%\bibliographystyle{alpha}
%\bibliographystyle{unsrt}
%\bibliographystyle{abbrv}
\bibliography{ProbEstThrmInfCtrl}

\newpage

%------------------------------------------------------------------------------

\appendix

%------------------------------------------------------------------------------
\end{document}
