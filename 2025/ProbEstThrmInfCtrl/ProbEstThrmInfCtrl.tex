% easychair.tex,v 3.5 2017/03/15

\documentclass{easychair}
%\documentclass[EPiC]{easychair}
%\documentclass[EPiCempty]{easychair}
%\documentclass[debug]{easychair}
%\documentclass[verbose]{easychair}
%\documentclass[notimes]{easychair}
%\documentclass[withtimes]{easychair}
%\documentclass[a4paper]{easychair}
%\documentclass[letterpaper]{easychair}

\usepackage{doc}
\usepackage{minted}
%% \usepackage{amsfonts}
\usepackage{graphicx}
\usepackage{amsmath}
\usepackage{amssymb}
\usepackage{bussproofs}
\usepackage{cite}

% For ≞ (requires the LuaLaTeX engine)
\usepackage{unicode-math}
\setmathfont{Stix Two Math}

% Use this if you have a long article and want to create an index
% \usepackage{makeidx}

% In order to save space or manage large tables or figures in a
% landcape-like text, you can use the rotating and pdflscape
% packages. Uncomment the desired from the below.
%
% \usepackage{rotating}
% \usepackage{pdflscape}

% Some of our commands for this guide.
%
\newcommand{\easychair}{\textsf{easychair}}
\newcommand{\miktex}{MiK{\TeX}}
\newcommand{\texniccenter}{{\TeX}nicCenter}
\newcommand{\makefile}{\texttt{Makefile}}
\newcommand{\latexeditor}{LEd}
\newcommand{\U}{\Theta}
\newcommand{\Theory}{\texttt{Theory}}
\newcommand{\Proof}{\texttt{Proof}}
\newcommand{\Proposition}{\texttt{Proposition}}
\newcommand{\Bool}{\texttt{Bool}}
\newcommand{\arrow}{\to}
\newcommand{\limp}{\Rightarrow}
\newcommand{\True}{\texttt{True}}
\newcommand{\False}{\texttt{False}}
\newcommand{\STV}[2]{<\!#1, #2\!>}

%\makeindex

%% Front Matter
%%
% Regular title as in the article class.
%
%% \title{Estimating the Probability of a Proposition to be Provable
%%   Using Probabilistic Logic Networks}
\title{Estimating the Probability of a Conjecture to Be a Theorem with
  Probabilistic Logic Networks, and How to Take Advantage of it for
  Inference Control}

% Authors are joined by \and. Their affiliations are given by \inst, which indexes
% into the list defined using \institute
%
\author{Nil Geisweiller}

% Institutes for affiliations are also joined by \and,
\institute{
  SingularityNET Foundation,\\
  Zug, Switzerland\\
  \email{nil@singularitynet.io}
}

%  \authorrunning{} has to be set for the shorter version of the authors' names;
% otherwise a warning will be rendered in the running heads. When processed by
% EasyChair, this command is mandatory: a document without \authorrunning
% will be rejected by EasyChair
\authorrunning{Geisweiller}

% \titlerunning{} has to be set to either the main title or its shorter
% version for the running heads. When processed by
% EasyChair, this command is mandatory: a document without \titlerunning
% will be rejected by EasyChair
\titlerunning{Estimating the Probability of a Conjecture to Be a Theorem
  with Probabilistic Logic Networks}

\begin{document}

\maketitle

%% \begin{abstract}
%% \end{abstract}

\section{Introduction}

In this paper we will show how to estimate the probability of a
proposition to be provable given all available evidence by using
Probabilistic Logic Networks (PLN)~\cite{Goertzel09PLN}.  We will then
explain how such estimation can be used as guiding heuristics for
Automated Theorem Proving (ATP).

The idea is to define a ternary predicate holding the relationship
between theory, proof and theorem.  Given its semi-decidable nature,
under finite resources, we cannot hope to establish whether such
relationship hold for any triple of theory, proof and theorem.  We can
however hope to estimate, with various degrees of confidence, the
probability that it may or may not hold given the available evidence.
For instance, a piece of evidence in favor of
$$\forall x \ P(x)$$ could be that $P$ holds for some $a$.
Potentially every piece of information that relates to the conjecture
should be taken into account to estimate the probability that it is a
theorem.  Although only having an actual proof, or an actual
contradiction, may ever shift from probability to certainty.
Moreover, such ability can then be used as a guide to discover proofs
by prioritizing the search over lemmas that are themselves more likely
to be provable.  The same idea can be applied on these lemmas till the
recursion hopefully bottoms out by reaching the axioms or a
contradiction.

NEXT: add quick state of the art.

\section{Relating Theory, Proof and Theorem in PLN}

If reasoning is considered from a Type Theoretic angle, propositions
are types, theories are collections of typing relationships and proofs
are terms inhabiting types.  Let us define a ternary predicate $\U$,
representing such relationship
$$\U : \Theory \times \Proof \times \Proposition \arrow \Bool$$ where
$\Theory$ is a set of collections of typing relationships encoding the
axioms and inference rules of each theory, $\Proof$ is a set of terms
representing proofs, and $\Proposition$ is a set of types representing
propositions.  The content of $\U$ can already be characterized with
PLN statements expressing generic rewriting laws.  For instance modus
ponens could be represented as\footnote{Note that due to PLN being in
a state of rework, the syntax used here is provisional.}
$$\U(\Gamma, f, a \to b) \land \U(\Gamma, x, a) \limp \U(\Gamma, f(x),
b)\ \measeq\ \STV{1}{1}$$ where $\Gamma$, $f$, $x$, $a$ and $b$ are
universally quantified variables,  %% The symbol
%% $\measeq$ is proper to PLN and can be read as \emph{measured by}.
$\measeq$ relates a PLN \emph{statement}, here an implication, to a
\emph{true value}, here $\STV{1}{1}$, forming a PLN \emph{judgment}
capturing the uncertainty of the statement.  The first number of the
truth value represents the strength and the second number represents
the confidence over that strength, although underneath, a truth value
represents a second order distribution.  In the example above the
judgment is certain because both the strength and the confidence are
1.  The full judgment can be read as: in theory $\Gamma$, if $f$ is a
proof of $a \arrow b$, and $x$ is a proof of $a$, then with certainty
$f(x)$ is a proof of $b$.  In addition, PLN allows to reason about
uncertain knowledge via induction and abduction.  Given a corpus of
examples (and counter examples) of triples over $\Theta$, induction
could for instance be used to gather statistics about the probability
of any proposition to be a theorem, or more specifically the
probability of some proposition meeting some criterion to be a
theorem.

Given a theory $\Gamma$ and a proposition $C$, the question \emph{what
is the probability that there exists a proof $p$ of $C$ in $\Gamma$?}
can be formulated by the following PLN query
$$\exists p\ \U(\Gamma, p, C)\ \measeq\ \$\texttt{tv}$$ where
$\$\texttt{tv}$ is a hole corresponding to the truth value to be
filled by PLN.  The way this truth value would be calculated would
involve both crisp logical reasoning and statistical reasoning, the
latter including recognizing patterns relating theories, proofs and
propositions.
%% Thus given
%% $$\U(\Gamma, pa, P(a))\ \measeq\ \True$$
%% $$\U(\Gamma, pb, P(b))\ \measeq\ \True$$
%% $$\vdots$$
%% one can infer
%% $$\U(\Gamma, p, \forall x\ P(x))\ \measeq\ \True$$

\section{Provability Estimation as Guiding Heuristic}




Even though it looks like solving a problem (i.e. proving theorems in
crisp theories) a by having to solve even bigger problems
(i.e. proving theorems in PLN), PLN is part of $\Theta$.

\newpage

%------------------------------------------------------------------------------

\bibliographystyle{plain}
\label{sect:bib}
%\bibliographystyle{alpha}
%\bibliographystyle{unsrt}
%\bibliographystyle{abbrv}
\bibliography{ProbEstThrmInfCtrl}

\newpage

%------------------------------------------------------------------------------

\appendix

%------------------------------------------------------------------------------
\end{document}
