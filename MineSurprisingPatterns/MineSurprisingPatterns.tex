% This is samplepaper.tex, a sample chapter demonstrating the
% LLNCS macro package for Springer Computer Science proceedings;
% Version 2.20 of 2017/10/04
%
\documentclass[runningheads]{llncs}
%
\usepackage{graphicx}
\usepackage{amsmath}
% Used for displaying a sample figure. If possible, figure files should
% be included in EPS format.
%
% If you use the hyperref package, please uncomment the following line
% to display URLs in blue roman font according to Springer's eBook style:
% \renewcommand\UrlFont{\color{blue}\rmfamily}

\begin{document}
%
\title{Surprising Pattern Mining in Hypergraph as a Form of Reasoning}
%
%\titlerunning{Abbreviated paper title}
% If the paper title is too long for the running head, you can set
% an abbreviated paper title here
%
\author{Nil Geisweiller\inst{1}\orcidID{0000-1111-2222-3333} \and
Ben Goertzel\inst{2,3}\orcidID{1111-2222-3333-4444}}
%
\authorrunning{N. Geisweiller et al.}
% First names are abbreviated in the running head.
% If there are more than two authors, 'et al.' is used.
%
\institute{SingularityNET}
%
\maketitle              % typeset the header of the contribution
%

\begin{abstract}
  In this paper we introduce a pattern miner algorithm alongside a
  definition of surprisingness. The algorithm is framed as a reasoning
  process, implemented on top the Unified Rule Engine, OpenCog's
  rewriting system for carrying mathematical and common sense
  reasoning. Doing so confer advantages of transparency and
  flexibility of control. Some initial experiments are conducted on
  the SUMO ontology.  \keywords{Pattern Miner \and Surprisingness \and
    Reasoning.}
\end{abstract}

\section{Introduction}

Finding patterns in data is an important part of learning.

By placing more learning into reasoning it allows to

\begin{enumerate}
\item Leverage the existing inference control mechanism of OpenCog (to be
described in subsequent publications, but that follows the abstract
principles described in [REF: growth of pattern, etc].
\item Offers more transparency, both for the human and the machine,
  bringing a form of XAI (eXplanatory AI).
\end{enumerate}
Although such a reframing carries a computational cost it confers a
significant benefit, which is that the search over the space of
patterns, by far the most costly aspect of pattern mining is amenable
to meta-learning [REF].

Pattern mining itself, if considered in its broadest form can be a
very expensive process. Implementing it on top of the Unified Rule
Engine confer multiple advantages.

\section{Mining Surprising Patterns in Hypergraph Data Base}

\subsection{Patterns as Queries}

In OpenCog, data can be retrieved from the atomspace via programs
describing queries, similar in spirit to SQL queries, represented
themselves as hypergraphs! This insure complete reflexivity of the
querying language, queries can be queried, produced, inferred, etc.

What the pattern miner does is essentially to evolve such program
queries, which is underneath also an evolution of inferences.

Here's an example of such query programs (provided in Scheme syntax,
the primary binding language for OpenCog)
\begin{verbatim}
(Get
  (Present
    (Inheritance (Variable "$X") (Variable "$Y"))
    (Inheritance (Variable "$Y") (Variable "$Z"))))
\end{verbatim}
which fetch any instance of transitivity in the AtomSpace. For
instance if the AtomSpace contains
\begin{verbatim}
(Inheritance (Concept "cat") (Concept "mammal"))
(Inheritance (Concept "mammal") (Concept "animal"))
(Inheritance (Concept "square") (Concept "geometric-shape"))
\end{verbatim}
it will retrieve
\begin{verbatim}
(Set
  (List (Concept "cat") (Concept "mammal") (Concept "animal")))
\end{verbatim}
where \texttt{cat}, \texttt{mammal} and \texttt{animal} are associated
to variable \texttt{\$X}, \texttt{\$Y} and \texttt{\$Z} according to a
prefix order. \texttt{square} and \texttt{geometric-shape} are not
retrieved because they do not exhibit transitivity in the data.

The construct \texttt{Get} means retrieve. The construct
\texttt{Present} means that the following expressions are patterns to
be conjunctively matched against the data present in the atomspace. We
also call the argument expressions of \texttt{Present} clauses and say
that the whole pattern is a conjunction of clauses.

The component that takes a query and returns the matching data is
called the \emph{pattern matcher}. The pattern miner solves the
inverse problem of pattern matcher. That is provided data, it attempts
to find queries that would retrieve a certain minimum number of
matches. This number corresponds to the \emph{support} in the pattern
mining terminology.

More constructs exist to build sophisticated queries such as type
declarations, preconditions, and pretty much any computation that one
may need to express. However the pattern miner only supports a subset
of these, due to the inherent complexity of dealing with such
expressiveness. The type of queries like the example above are
supported and already allows to capture a wide range of patterns. For
instance, as the reader would have probably noticed, clauses can be
connected via joint variables. In the example above such a joint
variable is \texttt{\$Y}.

\subsection{Pattern Mining High Level Algorithm}

Frequent Subtree Mining \cite{Chi2005Freq} refers to a group
techniques to discover patterns within a data base of trees.

Patterns are Atomese programs, programs expressing queries, akin to
SQL queries.

To discover surprising patterns we first mine frequent patterns,
pruning the search using the apriori property
\cite{Agrawal1994fastalgorithms}, as well as some home brewed
heuristics. Then we calculate the surprisingness of all resulting
patterns.

Much of the terminology used in this paper is borrowed from
. Let us recall the most important terms
\begin{itemize}
\item \emph{Text tree}: a tree, here a hypergraph, that is part of the
  data set to mine. Can be simply called \emph{tree}. Generally
  speaking any atom of an atomspace.
\item \emph{Pattern tree}: a tree representing a pattern, that is
  capturing a collection of text trees. Can be simply called
  \emph{pattern}.
\item \emph{Count}: number of text trees and subtrees matching a
  given pattern.
\item \emph{Frequency}: count of a pattern divided by the size of the
  universe, here the number of nodes of the hypergraph.
\item \emph{Support}: similar to count.
\item \emph{Minimum support}: parameter of the mining algorithm to discard
  patterns with frequency below that value.
\item \emph{Apriori property}: assumption that allows to systematically prune
  the search space. In its least abstract form, it expresses the fact
  that if a pattern tree has a certain frequency `f` then a
  specialization of it can only have a frequency that is equal to or
  lower than `f`.
\end{itemize}

Pattern mining operates by searching the space of patterns, typically
starting from the most abstract pattern, the \emph{top} pattern that
encompasses all trees, then constructing specializations of it, retain
those that have enough support (frequency equal to or above the
threshold), recursively specializing those, and so on.

Given a data base $\mathcal{D}$, a minimal support $S$ and an initialize
collection $\mathcal{C}$ of patterns with their counts equal to or above $S$,
the mining algorithm is as follows
\begin{enumerate}
\item select a pattern $P$ from $\mathcal{C}$,
\item find all shallow specializations $\mathcal{E}$ of $P$ with
  support equal to or above $S$,
\item remove $P$ from $\mathcal{C}$ and add the new patterns
  $\mathcal{E}$ to $\mathcal{C}$,
\item repeat till a termination criterion is met (maximum number of
  iterations, etc).
\end{enumerate}

The pattern collection $\mathcal{C}$ is usually initialized with the
top pattern, from which all subsequent pattern derive from.

\section{Framing Pattern Mining as Reasoning}

The tricky part of the algorithm above is step 1, selecting which
pattern to expand at each iteration, which will determine how the
space of patterns is being searched. However by framing pattern mining
as reasoning we can.

We will just mention the fact that due to having the search framed as
a reasoning process, the following step actually amounts to selecting
a source in a forward chaining reasoning.

The type of propositions we need to prove have the following forms
$$ S \le \text{support}(P, \mathcal{D}) $$ expressing that P has
enough support with respect to the data base $\mathcal{D}$, or simply
$$ \text{minsup}(P, S, \mathcal{D}) $$ where $\text{minsup}$ is
defined as
$$ \text{minsup}(P, S, \mathcal{D}) := S \le \text{support}(P,
\mathcal{D}) $$

The type of knowledge above can be inferred with the rule
$$ \text{minsup}(Q, S, \mathcal{D}) \land \text{spec}(Q, P) \vdash
\text{minsup}(P, S, \mathcal{D})$$ expressing that if $Q$ has enough
support, and $Q$ is a specialization of $P$, then $P$ has enough
support, essentially formalizing the Apriori property\footnote{or
  rather its simplest variation}.

Other rules can be used to infer knowledge about $\text{minsup}$, more
considered as heuristics as, unlike the apriori property, they do not
guaranty completeness, but can speed-up the search. For instance the
following rule
$$ \text{minsup}(P, S, \mathcal{D}) \land \text{minsup}(Q, S,
\mathcal{D}) \land \mathcal{P}(P \otimes Q) \vdash \text{minsup}(P
\otimes Q, S, \mathcal{D}) $$ expresses that if $P$ and $Q$ have
enough support, and a certain combination of $P$ and $Q$ has a certain
property, then such combination has enough support. An example of such
rule will be given below.

\section{Surprisingness}

Even with the help of the apriori property and additional heuristics
to prune the search, the volume of found patterns can still be
overwhelming. For that it can be helpful to assign to the patterns a
measure of \emph{interestingness}. Such notion is broad and can mean
various things in various contexts. We will restrict our attention on
the sub-notion of \emph{surprisingness}. Although still broad one can
define surprisingness as what is \emph{contrary to expectations}. It
is not too difficult to see, in the context of discovering patterns,
why such notion is important. If some pattern contradicts expectations
it means we have uncovered an incorrectness or incompleteness in our
model of the world and thus requires special attention.

Just like for pattern mining, surprisingness can be framed as
reasoning. They are many ways to formalize such notion. We suggest
that in its most general sense, surpisingness may be the difference of
outcome between different inferences over the same conjecture. Of
course in most logical systems, if consistent, different inferences
will produce the same result. However para-consistent systems, such as
PLN \cite{Goertzel2009PLN} for \emph{Probabilistic Logic Network},
OpenCog's logic for common sense reasoning, allow different inferences
to produce different outcomes on the a given conjecture. In particular
PLN allows a conjecture to be believed with various degrees of truth,
ranging from total ignorance to absolute certainty. Thus PLN is
particularly well suited for such definition of surprisingness. More
specifically we define surprisingness as the \emph{distance of truth
  values between different inferences over the same conjecture}. In
PLN a truth value is a second order distribution, probabilities over
probabilities, as explained in Chapter 4 of
\cite{Goertzel2009PLN}. Second order distributions allow to capture
uncertainties. Total ignorance is represented by a flat distribution
(Bayesian prior), or possible a slightly concave one (Jeffreys prior
\cite{Jeffreys46Invariant}), and absolute certainty by a Dirac delta
function.

Such definition of surprisingness has the merit of encompassing a wide
variety of cases; like the surprisingness of finding a proof
contradicting human intuition. For instance the outcome of Euclid's
proof of the infinity of prime numbers might contradict the intuition
of a beginner upon observation that prime numbers rapidly rarefy as
numbers grow. It also encompasses the surprisingness of observing an
unexpected event, or the surprisingness of discovering a pattern in
seemingly random data. All these cases can be framed as ways of
constructing different types of inferences and finding contradictions
between them. For instance in the case of discovering a pattern in a
data set, one inference could calculate the empirical probability
based on the data, while an other inference could calculate a
probability estimate based on independence of the variables and their
marginal probabilities.

To be closer to the human notion of surprisingness one may refine this
definition by having one type of inference as being the current model
of the world from which rapid (and usually uncertain) conclusions can
be derived, and the other type of inference provided by the world
itself, either via observations, in the case of an experienced
reality, or via crisp and long chains of deductions in the case of a
mathematical reality.

The distance measure to use to compare conjecture outcomes remains to
be defined. Since our truth values are distributions the
Jensen-Shannon distance \cite{Endres2003A}, also suggested as
surprisingness measure in \cite{Pienta2015AN} and
\cite{DerezinskiRH18}, could be used. The advantage of such distance
is that it accounts well for uncertainty. If for instance a pattern is
discovered in a small data set displaying high levels of dependencies
between variables (thus surprising relative to an independence
assumption of these variables), the surprisingness measure should
consider the possibility that it might be a fluke since the data set
is small. Fortunately, the smaller the data set, the flatter the
second order distributions representing the empirical and the
estimated truth values of the pattern, automatically reducing the
Jensen-Shannon distance.

Likewise one can imagine the following coin tossing experiments. In
the first experiment a coin is tossed 3 times, a probability $p_1$ of
head is calculated, then the coin is tossed 3 more times, a second
probability $p_2$ of head is calculated. $p_1$ and $p_2$ might be very
different, but it should not be surprising given the low number of
observations. On the contrary, in the second experiment the coin is
tossed a billion times, $p_1$ is calculated, then another billion
times and $p_2$ is calculated. Here even tiny differences between
$p_1$ and $p_2$ should be considered surprising. In both cases, by
representing $p_1$ and $p_2$ as second order distributions rather than
mere probabilities, the Jensen-Shannon distance properly accounts for
the uncertainty.

From now on, for sake of brevity, when speaking of \emph{probability}
we actually mean \emph{second order probability distribution}. When
the probability is estimated by a Bernoulli process (which is the case
in most of the remaining uses) the second order probability
distribution can easily be obtained by considering a Beta-distribution
[REF].

\subsection{Independence-based Surprisingness}

Here we explore a simple form of surprisingness based on the
independence of the clauses (not variables) within a pattern called
I-Surprisingness for Independence-Surprisingness. For instance
\begin{verbatim}
(Get
  (Present
    (Inheritance (Variable "$X") (Variable "$Y"))
    (Inheritance (Variable "$Y") (Variable "$Z"))))
\end{verbatim}
has two clauses
\begin{verbatim}
(Inheritance (Variable "$X") (Variable "$Y"))
\end{verbatim}
and
\begin{verbatim}
(Inheritance (Variable "$Y") (Variable "$Z"))
\end{verbatim}
joint by \texttt{\$Y}. Each clause considered independently, one can
use the data to calculate the empirical probability of the
clause. That is, considering the first clause, the probability of
having an assignment (\texttt{\$X=x},\texttt{\$Y=y}) such that
\texttt{(Inheritance x y)} appears in the data.

The goal is to, given such empirical probabilities, calculate the
probability estimate of their conjunctions. The presence of joint
variables makes it impossible to merely calculate the estimate of the
conjunction as the product of their probabilities as the joint
variables need to be accounted for. To do that we use the fact that a
pattern of connected clauses is equivalent to a pattern of
disconnected clauses combined with a condition of equality between the
joint variables. For instance
\begin{verbatim}
(Get
  (Present
    (Inheritance (Variable "$X") (Variable "$Y"))
    (Inheritance (Variable "$Y") (Variable "$Z"))))
\end{verbatim}
is equivalent to
\begin{verbatim}
(Get
  (And
    (Present
      (Inheritance (Variable "$X") (Variable "$Y1"))
      (Inheritance (Variable "$Y2") (Variable "$Z")))
    (Equal (Variable "$Y1") (Variable "$Y2"))))
\end{verbatim}
where the joint variables, here \texttt{\$Y}, have been replaced by
variable occurrences in each clause, \texttt{\$Y1} and
\texttt{\$Y2}. This allows us to express the probability estimate as
the product of the probabilities of the clauses times the
probabilities of having equal values of assigned to the variables
occurrences, called $P_e$.

A excellent estimate of $P_e$ can be obtained as the inner product of
the distributions over all values that can be taken by the variable
occurrences. However such inner product does capture the interactions
between the clauses happening at the joint variables and thus
contradict the assumption of independence. It is ``too good'' for
I-surprisingness. We initially tried that and almost nothing was
measure as surprising. Instead we need to estimate $P_e$ without using
any empirical knowledge about the interactions if the clauses. For
that we can simply estimate the probability of two variable
occurrences being equal be considering the number of possible values
they can each take. Let us denote $V(X_i)$ the set of values that
variable occurrence $X_i$ can take when its corresponding clauses is
considered independently, and $|V(X_i)|$ the size of $V(X_i)$. Thus,
assuming $n$ clauses connected by variable $X$, an estimate would be

$$
P_e(X_1=...=X_n)~=\prod_{i=2}^n |V(X_i)|^{-1}
$$

assuming that all values in $|V(X_i)|$ are distributed uniformly. The
product ignores one of variable occurrence because for each value we
want to estimate the probability that the \emph{other} values equal to
this one. Then the quality of the estimate depends on the quality of
the estimations of $|V(X_i)|$. We will not enter in too much details
but what we have found is that by considering specialization
relationships (captured by the \texttt{spec} predicate defined in
[REF]) relative to joint variables between the clauses, it allows us
to place an upper bound on the number values $|V(X_i)|$ can take, and
provides an excellent estimate, yet without having to utilize the
interactions between the clauses that are hidden in the data.

In the end we obtain the following inference rule to calculate the
estimate

$$
P_e(X_1=...=X_n) = Prod_{i=2}^n M(X_i)^{-1}
$$

where $M(X_i)$ is either
\begin{enumerate}
\item $|V(X_j)|$, the number of values that $X_j$ can take in the most
  specialized clause where $X_j$ occurs that is either more abstract
  than or equivalent to the clauses where $X_i$ appears relative to
  $X$ and such that j<i.
\item $|U|$, the size of the data base if no such more abstract or
  equivalent component exists for $X_i$.
\end{enumerate}

This allows to see how such calculation can be entirely expressed as a
reasoning process, although in practice it is wrapped into a single
inference rule for efficiency purpose.

\subsection{Beyond Independence-based Surprisingness and Dynamic Measure}

Surprisingness has to be dynamic be nature, once a fact is known it is
no longer surprising, thus...

The advantage of framing measuring surprisingness as a reasoning
process is that we can easily make the transition from a hard coded
definition of surprisingness to a more general one. 

\section{Experiment: Mining Surprising Patterns in SUMO}

We have experiemented our current surprisingness measure on
synthesized data (to verify the correctness of the algorithm described
in REF), as well as real world data, in particular on the SUMO
ontology REF. Most surprising patterns discovered are rather abstract
and thus difficult to interpret, in spite of being surprising. Such
patterns are for instance

TODO

Also patterns that are concrete enough to be understood by a human,
such as

TODO (maritim)

are usually not surprising to a human because, although they are
indeed distant from their probability estimate under independence
assumptions, thus considered as surprising to the machine, are not
surprising to us because we can use our background knowledge of the
world as well as our capacity to infer a better estimate to be able to
not be surprised.

\section{Conclusion}

It is clear from that experiment that the next step is to enable some
form of open-ended (yet reasonably fast) means of inferences as to
take into account more semantic knowledge, and thus .

%
% ---- Bibliography ----
%
\bibliographystyle{splncs04}
\bibliography{my}

\end{document}
