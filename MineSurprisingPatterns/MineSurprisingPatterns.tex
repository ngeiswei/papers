% This is samplepaper.tex, a sample chapter demonstrating the
% LLNCS macro package for Springer Computer Science proceedings;
% Version 2.20 of 2017/10/04
%
\documentclass[runningheads]{llncs}
%
\usepackage{graphicx}
\usepackage{amsmath}
\usepackage{bussproofs}
\usepackage{cite}
% Used for displaying a sample figure. If possible, figure files should
% be included in EPS format.
%
% If you use the hyperref package, please uncomment the following line
% to display URLs in blue roman font according to Springer's eBook style:
% \renewcommand\UrlFont{\color{blue}\rmfamily}

\begin{document}
%
\title{An Inferential Approach to Mining Surprising Patterns in Hypergraphs}
%
%\titlerunning{Abbreviated paper title}
% If the paper title is too long for the running head, you can set
% an abbreviated paper title here
%
\author{Nil Geisweiller
  %\orcidID{0000-0001-5041-6299}
  \and Ben Goertzel}
%
\authorrunning{N. Geisweiller et al.}
% First names are abbreviated in the running head.
% If there are more than two authors, 'et al.' is used.
%
\institute{ SingularityNET Foundation, The
  Netherlands\\ \email{\{nil,ben\}@singularitynet.io}}
%
\maketitle              % typeset the header of the contribution
%

\begin{abstract}
  A novel pattern mining algorithm and a novel formal
  definition of surprisingness are introduced, both framed in
  the context of formal reasoning.   Hypergraphs are used 
  to represent the data in which patterns are mined, the patterns
  themselves, and the control rules for the pattern miner.
   The implementation of these
  tools in the OpenCog framework, as part of a broader
  multi-algorithm approach to AGI, is described.

  \keywords{Pattern Miner \and Surprisingness \and Reasoning \and Hypergraphs.}
\end{abstract}

\section{Introduction}

Pattern recognition is broadly recognized as a key aspect
of general intelligence, as well as of many varieties of
specialized intelligence.   Finding patterns is a way of discovering knowledge and learning about
the data the world they represents. These patterns can then be used
in various ways like constructing predictive models to help agents act in
this world \cite{Jade12Pat}, or they can be passed on to human
expertise.  General intelligence can be envisioned, among other ways, as the process
of an autonomous agent recognizing patterns in itself and its environment,
including patterns regarding which of its actions tend to achieve which goals in
which contexts \cite{Goertzel2014EGI1}.

The scope of pattern recognition algorithms in AI and allied disciplines
is very broad, including many specialized algorithms aimed at recognizing
patterns in particular sorts of data such as visual data, auditory data or
genomic data.   Among more general-purpose approaches to pattern recognition,
so-called "pattern mining" plays a prominent role.   Mining here refers to the process
of systematically searching a body of data to find a large number of patterns
satisfying certain criteria.   Most pattern mining algorithms are greedy in operation,
meaning they start by finding simple patterns and then try to combine these to guide
their search for more complex patterns, and iterate this approach a few times.

Pattern mining algorithms tend to work at the
\emph{syntactic} level, such as subtree mining \cite{Chi2005Freq},
where patterns are subtrees within a database of trees, and each
subtree represents a concept containing all the trees compatible with
that subtree. This is both a limit and a strength. Limit because they
cannot express arbitrary abstractions, and strength because for that
reason they can be relatively efficient. Moreover even purely
syntactic pattern miners can go a long way if much of the semantic
knowledge is turned into syntax. For instance if the data contains
$$\texttt{human(John)}$$
$$\texttt{human}\Rightarrow\texttt{mortal}$$ a purely syntactic
pattern miner will not be able to take into account the implicit
datum $$\texttt{mortal(John)}$$ unless a step of inference is formerly
taken to make it visible.

Another shortcoming of pattern mining is the volume of patterns it
tends to produce. For that reason it can be useful to rank the patterns found
according to some measure of interestingness \cite{Vreeken2014}.  One can
also use pattern mining in combination with other pattern recognition techniques,
e.g. evolutionary programming or logical inference.

Here we present a novel approach to pattern mining that combines semantic
with syntactic understanding of patterns, and that uses a sophisticated
measure of pattern surprisingness to filter the combinatorial explosion of
patterns.   The surprisingness measure and the semantic aspect of patterns
are handled via embedding the pattern mining process in an inference engine,
operating on a highly general hypergraph-based knowledge representation.

\subsection{Contribution}

We introduce a pattern miner algorithm designed to find patterns in
hypergraph database, alongside a measure of surprisingness designed
for use with this pattern miner. Both are implemented on the OpenCog
framework \cite{Goertzel2014EGI2}, on top of the \emph{Unified Rule
  Engine}, URE for short, the reasoning engine of OpenCog.

Framing pattern mining as reasoning provides the following
advantages:
\begin{enumerate}
\item Enable hybridizations between syntactic and semantic pattern
  mining.
\item Allow to handle the full notion of surprisingness, as will be
  further shown.
\item Offer more transparency. Produced knowledge can be reasoned
  upon. Reasoning steps selected during mining can be represented as
  data for subsequent mining and reasoning.
\item Enable meta-learning by leveraging URE's inference control
  mechanism.
\end{enumerate}
The last point, although already important as it stands, goes further
than it may at first seem. One of the motivations to have a pattern miner in OpenCog
is to mine inference traces, discover control rules out of them, and
apply these control rules to speed up reasoning in general. By framing
not only pattern mining but more generally learning as reasoning we
hope to kickstart a virtuous self-improvement cycle. Towards that end
more components of OpenCog, such as MOSES
\cite{Looks06abstractcompetent}, an evolutionary program learner, are
currently in the process of being ported to the URE as well.

Framing learning as reasoning is not without drawbacks, as more
transparency comes at a greater computational cost. However by
carefully partitioning transparent/costly versus opaque/efficient
computations we hope that an adequate balance between efficiency and
open-endedness can be achieved. For instance in the case of
evolutionary programming, decisions pertaining to what regions of the
program space to explore is best processed as reasoning, given the
importance and the cost of such operation. While more systematic
operations such as evaluating the fitness of a candidate can be left
as opaque. One may even draw a speculative analogy here with the distinction
between conscious and unconscious processes.

\subsection{Outline}

In Section \ref{PMHD} a pattern mining algorithm over hypergraphs is
presented; it is framed as reasoning in Section \ref{FPMR}. In Section
\ref{SUR} a definition of surprisingness is provided, and a more
specialized implementation is derived from it. Then, in Section
\ref{ISRB} an example of how it can be framed as reasoning is
presented, both for the specialized and abstract definitions of
surprisingness.

\section{Pattern Mining in Hypergraph Database}
\label{PMHD}

\subsection{AtomSpace: Hypergraph Database}

Let us first rapidly recall what is the AtomSpace
\cite{Goertzel2014EGI2}, the hypergraph knowledge store with which we
shall work here.  The AtomSpace is the OpenCog AGI framework's primary
data storage solution. It is a labeled hypergraph particularly suited
for representing symbolic knowledge, but is also capable of
representing sub-symbolic knowledge (probabilities, tensors, etc), and
most importantly combinations of the two. In the OpenCog terminology,
edges of that hypergraph are called \emph{links}, vertices are called
\emph{nodes}, and \emph{atoms} are either links or nodes.

For example
one may express that cars are vehicles with
\begin{verbatim}
(Inheritance (Concept "car") (Concept "vehicle"))
\end{verbatim}
\texttt{Inheritance} is a link connecting two concept nodes,
\texttt{car} and \texttt{vehicle}.  If one wishes to express the other
way around, how much vehicles are cars, then one can attach the
inheritance with a \emph{truth value}
\begin{verbatim}
(Inheritance (stv 0.4 0.8) (Concept "vehicle") (Concept "car"))
\end{verbatim}
where 0.4 represents a probability and 0.8 represents a confidence.

Storing knowledge as hypergraph rather than collections of formulae
allows to rapidly query atoms and how they relate to other atoms. For
instance given the node $\texttt{(Concept "car")}$ one can query the
links pointing to it, here two, one expressing that cars are vehicle,
the other expressing that some vehicles are cars.

\subsection{Pattern Matching}
\label{PM}

OpenCog comes with a \emph{pattern matcher}, a component that can
query the AtomSpace, similar in spirit to SQL, but different in
several aspects. For instance queries are themselves programs
represented as atoms in the AtomSpace. This insures reflexivity where
queries can be queried or produced by queries. 

Here's an example of such a query
\begin{verbatim}
(Get
  (Present
    (Inheritance (Variable "$X") (Variable "$Y"))
    (Inheritance (Variable "$Y") (Variable "$Z"))))
\end{verbatim}
which fetches any instance of transitivity of inheritance in the
AtomSpace. For instance if the AtomSpace contains
\begin{verbatim}
(Inheritance (Concept "cat") (Concept "mammal"))
(Inheritance (Concept "mammal") (Concept "animal"))
(Inheritance (Concept "square") (Concept "shape"))
\end{verbatim}
it retrieves
\begin{verbatim}
(Set
  (List (Concept "cat") (Concept "mammal") (Concept "animal")))
\end{verbatim}
where \texttt{cat}, \texttt{mammal} and \texttt{animal} are associated
to variable \texttt{\$X}, \texttt{\$Y} and \texttt{\$Z} according to
the prefix order of the query, but \texttt{square} and \texttt{shape}
are not retrieved because they do not exhibit transitivity. The
construct $\texttt{Set}$ represents a set of atoms, and
$\texttt{List}$ in this context represents tuples of values. The
construct \texttt{Get} means retrieve. The construct \texttt{Present}
means that the arguments are patterns to be conjunctively matched
against the data present in the AtomSpace. We also call
\texttt{Present} arguments \emph{clauses} and say that the pattern is
a \emph{conjunction of clauses}.

In addition, the pattern matcher can rewrite. For instance a
transitivity rule could be implemented with
\begin{verbatim}
(Bind
  (Present
    (Inheritance (Variable "$X") (Variable "$Y"))
    (Inheritance (Variable "$Y") (Variable "$Z")))
  (Inheritance (Variable "$X") (Variable "$Z")))
\end{verbatim}
The pattern matcher provides the building blocks for the reasoning
engine. In fact the URE is, for the most part, pattern matching +
unification. The collection of atoms that can be executed in OpenCog,
to query the atomspace, reason or such, forms a language called
\emph{Atomese}.

\subsection{Pattern Mining as Inverse of Pattern Matching}

The pattern miner solves the inverse problem of pattern matcher. That
is provided data, it attempts to find queries that would retrieve a
certain \emph{minimum} number of matches. This number is called the
\emph{support} in the pattern mining terminology \cite{Chi2005Freq,
  Agrawal1994fastalgorithms}.

It is worth mentioning that the pattern matcher has many more
constructs than \texttt{Get}, \texttt{Present} and \texttt{Bind}; for
declaring types, expressing preconditions, and implementing pretty
much any computation one may want to express. However the pattern
miner only supports a subset of constructs due to the inherent
complexity of dealing with such expressiveness.

\subsection{High Level Algorithm of the Pattern Miner}

Before showing how to express pattern mining as reasoning, let us
explain the algorithm itself.

Our pattern mining algorithm operates like most pattern mining
algorithms \cite{Chi2005Freq} by a greedy algorithm of
searching the space of frequent
patterns while pruning the parts that do not have enough support. It
typically starts from the most abstract one, the \emph{top} pattern,
constructing specializations of it and only retain those that have
enough support, then repeat. The apriori property
\cite{Agrawal1994fastalgorithms} guaranties that no pattern with
enough support will be missed based on the fact that patterns without
enough support cannot have specializations with enough support.

Let us semi-formalize the above. Given a database $\mathcal{D}$, a
minimal support $S$ and an initialize collection $\mathcal{C}$ of
patterns with enough support, the mining algorithm operates as follows
\begin{enumerate}
\item Select a pattern $P$ from $\mathcal{C}$.
\item Produce a \emph{shallow specialization} $Q$ of $P$ with support
  equal to or above $S$.
\item Add $Q$ to $\mathcal{C}$, remove $P$ if all its shallow
  specializations have been produced.
\item Repeat till a termination criterion is met.
\end{enumerate}
The pattern collection $\mathcal{C}$ is usually initialized with the
top pattern
\begin{verbatim}
(Get
  (Present
    (Variable "$X")))
\end{verbatim}
that matches the whole database, and from which all subsequent
patterns are specialized. A shallow specialization is a specialization
such that the expansion is only a level deep. For instance, if
$\mathcal{D}$ is the 3 inheritances links of Subsection \ref{PM} (cat
is a mammal, a mammal is an animal and square is a shape), a shallow
specialization of the top pattern could be
\begin{verbatim}
(Get
  (Present
    (Inheritance (Variable "$X") (Variable "$Y"))))
\end{verbatim}
which would match all inheritance links, thus have a support of 3. A
subsequent shallow specialization of it could be
\begin{verbatim}
(Get
  (Present
    (Inheritance (Concept "cat") (Variable "$Y"))))
\end{verbatim}
which would only match
\begin{verbatim}
(Inheritance (Concept "cat") (Concept "mammal"))
\end{verbatim}
and have a support of 1. So if the minimum support $S$ is 2, this one
would be discarded. In practice the algorithm is complemented by
heuristics to avoid exhaustive search, but that is the crux of it.

\section{Framing Pattern Mining as Reasoning}
\label{FPMR}

The hardest part of the algorithm above is step 1, selecting which
pattern to expand; this has the biggest impact on how the space is
explored. When pattern mining is framed as reasoning such decision
corresponds to a \emph{premise or conclusion selection}.

Let us formalize the type of propositions we need to prove in order to
search the space of patterns. For sake of conciseness we will use a
hybridization between mathematics and Atomese, it being understood that
all can be formalized in Atomese.  Given a database $\mathcal{D}$ and
a minimum support $S$ we want to instantiate and prove the following
theorems
$$ S \le \texttt{support}(P, \mathcal{D}) $$ expressing that pattern
$P$ has enough support with respect to the data base $\mathcal{D}$, or
simplify
$$ \texttt{minsup}(P, S, \mathcal{D}) := S \le \text{support}(P,
\mathcal{D}) $$ The primary inference rule we need is (given in
Gentzen style),
\begin{prooftree}
  \AxiomC{$\texttt{minsup}(Q, S, \mathcal{D})$}
  \AxiomC{$\texttt{spec}(Q, P)$}
  \RightLabel{(AP)}
  \BinaryInfC{$\texttt{minsup}(P, S, \mathcal{D})$}
\end{prooftree}
expressing that if $Q$ has enough support, and $Q$ is a specialization
of $P$, then $P$ has enough support, essentially formalizing the
apriori property (AP). We can either apply such rule in a forward way,
top-down, or in a backward way, bottom-up. If we search from more
abstract to more specialized we want to use it in a backward
way. Meaning the reasoning engine needs to choose $P$
(\emph{conclusion selection} from $\texttt{minsup}(P, S,
\mathcal{D})$) and then construct a specialization $Q$.  In practice
that rule is actually written backward so that choosing $P$ amounts to
a \emph{premise selection}, but is presented here this way for
expository purpose.  The definition of $\texttt{spec}$ is left out,
but it is merely a variation of the subtree relationship while
accounting for variables (the same variable may occur at different
places in the pattern).
%% As
%% one can see a pattern such as
%% \begin{verbatim}
%% (Get
%%   (Present
%%     (Inheritance (Variable "$X") (Variable "$Y"))))
%% \end{verbatim}
%% is different than
%% \begin{verbatim}
%% (Get
%%   (Present
%%     (Inheritance (Variable "$X") (Variable "$X"))))
%% \end{verbatim}
%% where $\texttt{\$Y}$ has been replace by $\texttt{\$X}$, and this
%% requires additional care.

Other \emph{heuristic} rules can be used to infer knowledge about
$\texttt{minsup}$. They are heuristics because unlike the apriori
property, they do not guaranty completeness, but can speed-up the
search by eliminating large portions of the search space. For instance
the following rule
\begin{prooftree}
  \AxiomC{$\texttt{minsup}(P, S, \mathcal{D})$}
  \AxiomC{$\texttt{minsup}(Q, S, \mathcal{D})$}
  \AxiomC{$R(P \otimes Q)$}
  \RightLabel{(CE)}
  \TrinaryInfC{$\texttt{minsup}(P \otimes Q, S, \mathcal{D})$}
\end{prooftree}
expresses that if $P$ and $Q$ have enough support, and a certain
combination $P\otimes Q$ has a certain property $R$, then such
combination has enough support. Such rule can be used to build the
conjunction of multiple patterns. For instance given $P$ and $Q$ both
equal to
\begin{verbatim}
(Get
  (Present
    (Inheritance (Variable "$X") (Variable "$Y"))))
\end{verbatim}
One can combine them to form
\begin{verbatim}
(Get
  (Present
    (Inheritance (Variable "$X") (Variable "$Y"))
    (Inheritance (Variable "$Y") (Variable "$Z"))))
\end{verbatim}
The property $R$ here is that both clauses must share at least one
joint variable and the combination must have its support above or
equal to the minimum threshold.

\section{Surprisingness}
\label{SUR}

Even with the help of the apriori property and additional heuristics
to prune the search, the volume of mined patterns can still be
overwhelming. For that it is helpful to assign to the patterns a
measure of \emph{interestingness}. This is a broad notion and we will
restrict our attention to the sub-notion of \emph{surprisingness},
that can be defined as what is \emph{contrary to expectations}.
%% It is
%% important as if some pattern contradicts expectations it means an
%% incorrectness or incompleteness in the model of reality needs to be
%% addressed.

Just like for pattern mining, surprisingness can be framed as
reasoning. They are many ways to formalize it.  We tentatively suggest that in its
most general sense, surprisingness may be the considered as the difference of outcome
between different inferences over the same conjecture. 

Of course in
most conventional logical systems, if consistent, different inferences will produce
the same result. However in para-consistent systems, such as PLN for
\emph{Probabilistic Logic Network} \cite{Goertzel2009PLN}, OpenCog's
logic for common sense reasoning, conflicting outcomes are
possible. In particular PLN allows a proprosition to be believed with
various degrees of truth, ranging from total ignorance to absolute
certainty. Thus PLN is well suited for such definition of
surprisingness. 

More specifically we define surprisingness as the
\emph{distance of truth values between different inferences over the
  same conjecture}. In PLN a \emph{truth value} is a second order
distribution, probabilities over probabilities, Chapter 4 of
\cite{Goertzel2009PLN}. Second order distributions are good at
capturing uncertainties. Total ignorance is represented by a flat
distribution (Bayesian prior), or possibly a slightly concave one
(Jeffreys prior \cite{Jeffreys46Invariant}), and absolute certainty by
a Dirac delta function.

Such definition of surprisingness has the merit of encompassing a wide
variety of cases; like the surprisingness of finding a proof
contradicting human intuition. For instance the outcome of Euclid's
proof of the infinity of prime numbers might contradict the intuition
of a beginner upon observation that prime numbers rapidly rarefy as
numbers grow. It also encompasses the surprisingness of observing an
unexpected event, or the surprisingness of discovering a pattern in
seemingly random data. All these cases can be framed as ways of
constructing different types of inferences and finding contradictions
between them. For instance in the case of discovering a pattern in a
database, one inference could calculate the empirical probability
based on the data, while an other inference could calculate a
probability estimate based on variable independences.
%% and
%% their marginal probabilities.

The distance measure to use to compare conjecture outcomes remains to
be defined. Since our truth values are distributions the
\emph{Jensen-Shannon Distance}, JSD for short \cite{Endres2003A},
suggested as surprisingness measure in \cite{Pienta2015AN},
% DerezinskiRH18},
could be used. The advantage of such distance is that it accounts well
for uncertainty. If for instance a pattern is discovered in a small
data set displaying high levels of dependencies between variables
(thus surprising relative to an independence assumption), the
surprisingness measure should consider the possibility that it might
be a fluke since the data set is small. Fortunately, the smaller the
data set, the flatter the second order distributions representing the
empirical and the estimated truth values of the pattern, reducing the
JSD.

Likewise one can imagine the following coin tossing experiments. In
the first experiment a coin is tossed 3 times, a probability $p_1$ of
head is calculated, then the coin is tossed 3 more times, a second
probability $p_2$ of head is calculated. $p_1$ and $p_2$ might be very
different, but it should not be surprising given the low number of
observations. On the contrary, in the second experiment the coin is
tossed a billion times, $p_1$ is calculated, then another billion
times, $p_2$ is calculated. Here even tiny differences between $p_1$
and $p_2$ should be surprising. In both cases, by representing $p_1$
and $p_2$ as second order distributions rather than mere
probabilities, the Jensen-Shannon Distance seems to adequatly accounts
for the uncertainty.

A slight refinement of our definition of surprisingness, probably
closer to human intuition, can be obtained by fixing one type of
inference provided by the current model of the world from which rapid
(and usually uncertain) conclusions can be derived, and the other type
of inference implied by the world itself, either via observations, in
the case of an experiential reality, or via crisp and long chains of
deductions in the case of a mathematical reality.
%% We will attempt to
%% model that refinement of surprisingness.

\subsection{Independence-based Surprisingness}
\label{IS}

Here we explore a limited form of surprisingness based on the
independence of the variables involved in the clauses of a pattern,
called I-Surprisingness for Independence-based Surprisingness. For
instance
\begin{verbatim}
(Get
  (Present
    (Inheritance (Variable "$X") (Variable "$Y"))
    (Inheritance (Variable "$Y") (Variable "$Z"))))
\end{verbatim}
has two clauses
\begin{verbatim}
(Inheritance (Variable "$X") (Variable "$Y"))
\end{verbatim}
and
\begin{verbatim}
(Inheritance (Variable "$Y") (Variable "$Z"))
\end{verbatim}
If each clause is considered independently, that is the distribution
of values taken by the variable tuples $(\texttt{\$X}, \texttt{\$Y})$
appearing in the first clause is independent from the distribution of
values taken by the variable tuples $(\texttt{\$Y}, \texttt{\$Z})$ in
the second clause, one can simply use the product of the two
probabilities to obtain an probability estimate of their
conjunctions. However the presence of joint variables, here
$\texttt{\$Y}$, makes this calculation wrong. Instead the connections
need to be taken into account. To do that we use the fact that a
pattern of connected clauses is equivalent to a pattern of
disconnected clauses combined with a condition of equality between the
joint variables. For instance
\begin{verbatim}
(Get
  (Present
    (Inheritance (Variable "$X") (Variable "$Y"))
    (Inheritance (Variable "$Y") (Variable "$Z"))))
\end{verbatim}
is equivalent to
\begin{verbatim}
(Get
  (And
    (Present
      (Inheritance (Variable "$X") (Variable "$Y1"))
      (Inheritance (Variable "$Y2") (Variable "$Z")))
    (Equal (Variable "$Y1") (Variable "$Y2"))))
\end{verbatim}
where the joint variables, here $\texttt{\$Y}$, have been replaced by
variable occurrences in each clause, $\texttt{\$Y1}$ and
$\texttt{\$Y2}$. The construct $\texttt{Equal}$ tests the equality of
values, and the construct $\texttt{And}$ is the logical
conjunction.
%% So the query means, find values matching these two
%% patterns, such that each value associated to $\texttt{\$Y1}$ is equal
%% to the value associated to $\texttt{\$Y2}$.
Then we can express the probability estimate as the product of the
probabilities of the clauses, as if they were independent, times the
probability of having the values of the joint variables equal. The
formula estimating the probability of equality is not detailed here,
but according to our preliminary experiments, a decent estimate can be
obtained by performing some syntactic analysis to take into account
specializations between clauses relative to their joint variables.
%% to estimate the number of value a particular variable occurrence
%% can take.  \footnote{More detail is provided in the code
%% documentation of
%% https://github.com/opencog/opencog/blob/master/opencog/learning/miner}.

\section{I-Surprisingness Framed as Reasoning and Beyond}
\label{ISRB}

The proposition to infer in order to calculate surprisingness is
defined as
$$
\texttt{surp}(P, \mathcal{D}, s)
$$ where $\texttt{surp}$ is a predicate relating the pattern $P$ and
the database $\mathcal{D}$ to its surprisingness $s$, defined as
$$
s := \texttt{dst}(\texttt{emp}(P,\mathcal{D}),\texttt{est}(P,\mathcal{D}))
$$ where $\texttt{dst}$ is the Jensen-Shannon distance, $\texttt{emp}$
is the empirical second order distribution of $P$, and $\texttt{est}$
its estimate.

The calculation of $\texttt{emp}(P, \mathcal{D})$ is easily handled by
a \emph{direct evaluation} rule that uses the support of $P$ and the
size of $\mathcal{D}$ to obtain the parameters of the
beta-binomial-distribution describing its second order probability.
However, the mean by which the estimate is calculated is let
unspecified. This is up to the reasoning engine to find an inference
path to calculate it. Below is an example of inference tree to
calculate $\texttt{surp}$ based on I-Surprisingness
\begin{prooftree}
  \AxiomC{$P$}
  \AxiomC{$\mathcal{D}$}

  \AxiomC{$P$}
  \AxiomC{$\mathcal{D}$}  
  \RightLabel{(DE)}
  \BinaryInfC{$\texttt{emp}(P,\mathcal{D})$}

  \AxiomC{$P$}
  \AxiomC{$\mathcal{D}$}  
  \RightLabel{(IS)}
  \BinaryInfC{$\texttt{est}(P,\mathcal{D})$}

  \RightLabel{(JSD)}
  \BinaryInfC{$\texttt{dst}(\texttt{emp}(P,\mathcal{D}),\texttt{est}(P,\mathcal{D}))$}  

  \RightLabel{(S)}
  \TrinaryInfC{$\texttt{surp}(P, \mathcal{D}, \texttt{dst}(\texttt{emp}(P,\mathcal{D}),\texttt{est}(P,\mathcal{D})))$}
\end{prooftree}
where
\begin{itemize}
\item (S) is a rule to construct the $\texttt{surp}$ predicate,
\item (JSD) is a rule to calculate the Jensen-Shannon Distance,
\item (DE) is the direct evaluation rule to calculate the empirical
  second order probability of $P$ according to $\mathcal{D}$,
\item (IS) is a rule to calculate the estimate of $P$ based on
  I-Surprisingness described in Section \ref{IS}.
\end{itemize}
That inference tree uses a single rule (IS) to calculate the
estimate. Most rules are complex, such as (JSD), and actually have the
heavy part of the calculation coded in C++ for maximum efficiency. So
all that the URE must do is put together such inference tree, which
can be done reasonably well given how much complexity is encapsulated
in the rules.

As of today we have only implemented (IS) for the estimate. In
general, however, we want to have more rules, and ultimately enough so
that the estimate can be inferred in an open-ended way. In such
scenario, the inference tree would look very similar to the one above,
with the difference that the (IS) rule would be replaced by a
combination of other rules. Such approach naturally leads to a dynamic
surprisingness measure. Indeed, inferring that some pattern is
I-Surprising requires to infer its empirical probability, and this
knowledge can be further utilized to infer estimates of related
patterns. For instance, if say an I-Surprising pattern is discovered
about pets and food. A pattern about cats and food might also be
measured as I-Surprising, however the fact that cat inherits pet may
lead to constructing an inference that estimates the combination of
cat and food based on the combination of pet and food, possibly
leading to a much better estimate, and thus decreasing the
surprisingness of that pattern.

\section{Discussion}

The ideas presented above have been implemented as open source C++
code in the OpenCog framework, and have been evaluated on some initial
test datasets, including a set of logical relationships drawn from the
SUMO ontology \cite{Pease2011}.  The results of this empirical
experimentation are omitted here for space reasons and will be posted
online as supplementary information.  These early experiments provide
tentative validation of the sensibleness of the approach presented:
using inference on a hypergraph based representation to carry out
pattern mining that weaves together semantics and syntax and is
directed toward a sophisticated version of surprisingness rather than
simpler objective functions like frequency.

Future work will explore applications
to a variety of practical datasets, including empirical data and logs from an inference engine; and
richer integration of these methods with more powerful but more expensive techniques such as
predicate logic inference and evolutionary learning.


%% some independence assumptions will be
%% defaulted new inference paths may be created from to produce better
%% estimates for other patterns, which would be then be inferred as less
%% surprising. For instance


%% including enough eventually  but in principle it could and should be prevents from letting
%% it open-ended It could be calculated using I-Surprisingness as
%% described above or via other types of inferences.


%% That is the URE calls that
%% rule after running the pattern miner, which itself calls the C++
%% code. This provides another example of how learning can be partitioned
%% into transparent reasoning and opaque computation.
%% $$
%% \text{surp}(P, \mathcal{D},
%% \text{dst}(\text{emp}(P,\mathcal{D}),\text{est}(P,\mathcal{D}))
%% $$

%% After discovering a few surprising patterns based on such independence
%% assumption, the data should be expected to contain such dependencies,
%% and subsequent estimates should take that into account when
%% calculating of the surprisingness of new patterns. For instance once
%% the empirical probability of a pattern is known, if such a pattern was
%% surprising such knowledge should be retained and utilized, so that any
%% other pattern that has its probability estimate rapidly derivable from
%% the previously surprising patterns (and other background knowledge),
%% and such that this estimate is good (the distance between the
%% empirical probability and the estimate is small) should not be
%% considered surprising.

%% For that the current I-Surprisingness rule needs to be decomposed so
%% that the estimate can be calculated in a separate inference, the
%% empirical probability in a separate inference, and the two compared in
%% a separated Jensen-Shannon distance step.

%% Let's denote this probability $P_e$. To estimate $P_e$ while avoiding
%% empirical knowledge about the interactions of the
%% clauses\footnote{$P_e$ can be estimated with the inner product of the
%%   distributions over the values the joint variable occurrences, but
%%   this captures interactions between the clauses exhibited in the
%%   data. We initially tried that and almost nothing was measured as
%%   surprising.} we estimate the probability of two variable occurrences
%% being equal by assuming values are uniformly distributed. Let us
%% denote $S(X_i)$ the number of values $X_i$ can take when its
%% corresponding clauses is matched in isolation. Thus, assuming $n$
%% clauses connected by variable $X$, an estimate would be
%% $$
%% P_e(X_1=...=X_n)~=\prod_{i=2}^n S(X_i)^{-1}
%% $$

%% The product ignores one of variable occurrence because for each value
%% we want to estimate the probability that the \emph{other} values equal
%% to this one. The quality of the estimate depends on the quality of the
%% estimations of $S(X_i)$. Without entering in too much details what we
%% have found is that by considering specialization relationships because
%% the clauses (captured by the \texttt{spec} predicate defined in
%% Section [REF]) we can place an upper bound on the number of values
%% that $S(X_i)$ can take, and provides a decent estimate.

%% In the end we obtain the following inference rule to calculate the
%% estimate
%% $$
%% P_e(X_1=...=X_n) = Prod_{i=2}^n M(X_i)^{-1}
%% $$
%% where $M(X_i)$ is either
%% \begin{enumerate}
%% \item $|V(X_j)|$, the number of values that $X_j$ can take in the most
%% specialized clause where $X_j$ occurs that is either more abstract
%% than or equivalent to the clauses where $X_i$ appears relative to $X$
%% and such that j<i.
%% \item $|U|$, the size of the data base if no such more abstract or
%%   equivalent component exists for $X_i$.
%% \end{enumerate}

%% This allows to see how such calculation can be entirely expressed as a
%% reasoning process, although in practice it is wrapped into a single
%% inference rule for efficiency purpose.

%% \subsection{Beyond Independence-based Surprisingness and Dynamic Measure}

%% Surprisingness has to be dynamic be nature, once a fact is known it is
%% no longer surprising, thus...

%% The advantage of framing measuring surprisingness as a reasoning
%% process is that we can easily make the transition from a hard coded
%% definition of surprisingness to a more general one. 

%% \section{Experiment: Mining Surprising Patterns in SUMO}

%% We have experiemented our current surprisingness measure on
%% synthesized data (to verify the correctness of the algorithm described
%% in REF), as well as real world data, in particular on the SUMO
%% ontology REF. Most surprising patterns discovered are rather abstract
%% and thus difficult to interpret, in spite of being surprising. Such
%% patterns are for instance

%% TODO

%% Also patterns that are concrete enough to be understood by a human,
%% such as

%% TODO (maritim)

%% are usually not surprising to a human because, although they are
%% indeed distant from their probability estimate under independence
%% assumptions, thus considered as surprising to the machine, are not
%% surprising to us because we can use our background knowledge of the
%% world as well as our capacity to infer a better estimate to be able to
%% not be surprised.

%% \section{Conclusion}

%% It is clear from that experiment that the next step is to enable some
%% form of open-ended (yet reasonably fast) means of inferences as to
%% take into account more semantic knowledge, and thus .

%
% ---- Bibliography ----
%
\bibliographystyle{splncs04}
\bibliography{my}

\end{document}
