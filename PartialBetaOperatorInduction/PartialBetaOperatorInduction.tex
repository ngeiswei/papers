% This is samplepaper.tex, a sample chapter demonstrating the
% LLNCS macro package for Springer Computer Science proceedings;
% Version 2.20 of 2017/10/04
%
\documentclass[runningheads]{llncs}
%
% \usepackage{graphicx}
% Used for displaying a sample figure. If possible, figure files should
% be included in EPS format.
%
% If you use the hyperref package, please uncomment the following line
% to display URLs in blue roman font according to Springer's eBook style:
% \renewcommand\UrlFont{\color{blue}\rmfamily}

\usepackage{amsmath}
\usepackage{algorithm}
\usepackage[noend]{algpseudocode}

\begin{document}
%
\title{Partial Operator Induction with Beta Distributions}
%
%\titlerunning{Abbreviated paper title}
% If the paper title is too long for the running head, you can set
% an abbreviated paper title here
%
\author{Nil Geisweiller\inst{1,2,3}}
%
\authorrunning{N. Geisweiller}
% First names are abbreviated in the running head.
% If there are more than two authors, 'et al.' is used.
%
\institute{SingularityNET Foundation\\ \and
OpenCog Foundation\\ \and Novamente LLC\\
% \email{lncs@springer.com}\\
% \url{http://www.springer.com/gp/computer-science/lncs} \and
% \email{\{abc,lncs\}@uni-heidelberg.de}
}
%
\maketitle              % typeset the header of the contribution
%
\begin{abstract}
  A specialization of Solomonoff Operator Induction, driven by OpenCog
  model representation but expected to be more broadly useful,
  considering partial operators described by Beta distributions is
  introduced. The problem of taking into account partial operators in
  the prediction estimate is presented. This problem turns out to be
  non-trivial. A simplistic solution with a heuristic to estimate the
  Kolomogorov complexity of completions of partial models is given.

\keywords{Solomonoff Operator Induction \and Beta Distribution \and
  Bayesian Averaging.}
\end{abstract}
%
%
%
\section{Introduction}
Rarely natural intelligent agents attempt to construct complete models
of their environment. Often time they compartmentalize their knowledge
into contextual rules and make use of them without worrying about the
details of the assumingly remote and irrelevant parts of the world.

This is typically how AGI Prime, aka OpenCog Prime, the AGI agent
implemented over the OpenCog framework may utilize knowledge
\cite{Goertzel15Speculative}. The models we are specifically targeting
here are conditional probabilities, or to be more precise probability
distributions over conditional probabilities, or \emph{second order}
conditional probabilities. Maintaining second order probabilities is
how OpenCog accounts for uncertainties \cite{Ikle08Probabilistic} and
by that properly manages weighting knowledge from heterogeneous
sources, balancing exploitation and exploration and so on.
% In OpenCog terms, they are implication links with truth
% values describing their second order distributions.

We will sometimes call these models, rules, understanding that they
actually represent second order conditional probabilities. Here are
some examples of rules
% with the same goal of predicting temperature rises
% (propositions are assumed to be crisp,
%for instance either the sun shines, or it doesn't)
\begin{enumerate}
\item If the sun shines, then the temperature rises
\item If the sun shines and there is no wind, then the temperature rises
\item If the sun shines and the agent is in a cave, then the
  temperature rises
\end{enumerate}

These 3 rules have different degrees of truth. The first one is often
true, the second is nearly always true and the last one is rarely
true. The traditional way to quantify these degrees of truth is to
assign probabilities. In practice though these probabilities are
unknown, and instead one may only assign probability estimates based
on limited evidence. Or, according to the OpenCog design, one may
assign distributions over probabilities, capturing their degree of
certainty. The wider the less certain, the narrower the more certain.

% When such second order probability distribution is obtained by
% observations, like from instances of sun shines and temperature rises,
% a Beta-binomial distribution can be used [REF].

% S = sun shines
% T = temperature rises
% N = no wind
% C = in a cave

% the following

% f1(T|S)
% f2(T|S, N)
% f3(T|S, C)

Once degrees of truth are properly represented, an agent should be
able to utilize these rules to predict and operate in its
environment. This raises a question. How to choose between rules?
% Or how to weight them?
Someone wanting to predict whether the temperature
will rise will have to make a choice. If one is in a cave, should
he/she follow the third rule? Why not the first one which is valid, or
assuming there is no wind, maybe the second?

Systematically picking the rule with the narrowest context (like being
in a cave) is not always right. Indeed, the narrower the context the
less evidence we have, the broader the uncertainty, the more prone to
overfitting such rule might be.

\subsection{Contribution}

In this paper we attempt to address this issue by adapting Solomonoff
Operator Induction \cite{Solomonoff08Three} for a special class of
operators representing such rules. These operators have two
particularities.  First, their outcomes are second order
probabilities, specifically Beta distributions. Second, they are
partial, that is they are only defined over a subset of observations,
the available observations meeting the conditions of a given rule. For
instance if the goal is to predict the consequences of some actions
taken in the context of riding bicycle. Rules capturing that context
and no broader will not be able to account for observations made in
excluded contexts, such as walking. This latter particularity turns
out to be very difficult to address, and the solution we offer is very
lacking but presented nevertheless as a start.

% The motivation for writing this paper is twofold. First, although our
% specialization of Solomonoff Universal Operator Induction is driven by
% the specificities of OpenCog knowledge representation, Beta
% distribution for evidence based relationships, we believe it is
% general enough to be of interest to the AGI community. Second, while
% doing so we have uncovered an interesting problem (combination of
% partial operators), with no obvious solution, largely under addressed
% by the community, yet important. Although the solution we provide is
% very limited, we hope that paper helps to raise awareness and motivate
% further research in that direction.

\subsection{Overview}

In Section \ref{recall} we briefly recall Solomonoff Operator
Induction, Beta distributions. In Section \ref{part-op} we introduce
our specialization of Solomonoff Operator Induction for partial
operators with Beta distributions. Finally in Section \ref{con} we
conclude and present some directions for further research.

% \subsection{Notes}

% There is a continuum between complete uncertainty (operator output
% follows some prior) and perfect prediction, such as estimating missing
% data using the existing operators. We develop the perfect prediction
% hypothesis because it is the simplest one.

\section{Recall}
\label{recall}

\subsection{Solomonoff Operator Induction}
\label{sol-op-ind}
Solomonoff Universal Operator Induction \cite{Solomonoff08Three} is a
general, parameter free induction method that has been shown to
theoretically converge to any true computable distribution.  It is a
special case of Bayesian Model Averaging \cite{Hoeting99bayesianmodel}
though is universal in the sense that the models across which the
averaging is taking place are Turing-complete.

Let us recall its formulation, using the same notations as in the
original paper of Solomonoff (Section 3.2 of
\cite{Solomonoff08Three}). Given a sequence of $n$ questions and
answers $(Q_i, A_i)_{i \in [1, n]}$, and a countable family of
operators $O^j$ (the superscript $j$ denotes the $j^{th}$ operator,
not the exponentiation) computing partial functions mapping pairs of
question and answer to probabilities, then one may estimate the
probability of the next answer $A_{n+1}$ given new question $Q_{n+1}$
as follows
\begin{equation}
  \label{sol}
  \hat{P}(A_{n+1}|Q_{n+1}) = \sum_j a_0^j \prod_{i=1}^{n+1} O^j(A_i|Q_i)
\end{equation}
% , that is $O^j(A_i|Q_i)$ is the
% probability of answer $A_i$ given question $Q_i$,
where $a_0^j$ is the prior of the $j^{th}$ operator (its probability
after zero observation). Using Hutter's convergence theorems to
arbitrary alphabets \cite{Hutter03Optimality} it can be shown that
such estimate rapidly converges to the true probability.

Let us rewrite this equation by making the prediction term and the
likelihood explicit
\begin{equation}
  \label{sol-eas}
\hat{P}(A_{n+1}|Q_{n+1}) = \sum_j a_0^j l^j O^j(A_{n+1}|Q_{n+1})
\end{equation}
where $l^j = \prod_{i=1}^{n} O^j(A_i|Q_i)$ is the likelihood, the
probability of the data given the $j^{th}$ operator.\\

\begin{remark}
In the remaining of the paper the superscript $j$ is always used to
denote the index of the $j^{th}$ operator. Sometimes, though in a
consistent manner, it is used as subscript. All other superscript
notations not using $j$ denote exponentiation.
\end{remark}

\subsection{Beta Distribution}

Beta distributions \cite{Abourizk94Fitting} are convenient to model
probability distributions over probabilities, i.e. second order
probabilities. In particular, given a prior over a probability $p$ of
some event, like a coin toss to head, defined by a Beta distribution,
and a sequence of experiments, like $n$ coin tosses, the posterior of
$p$ is still a Beta distribution. For that reason the Beta
distribution is called a \emph{conjugate prior} for the binomial
distribution.

Let us recall the probability density and cumulative distribution
functions of the Beta distribution as it will be useful later on.
\subsubsection{Prior and Posterior Probability Density Function}
The probability density function (pdf) of the Beta distribution with
parameters $\alpha$ and $\beta$, is
\begin{equation}
  \label{beta-pdf}
f(x; \alpha, \beta) = \frac{x^{\alpha - 1} (1-x)^{\beta - 1}}
                           {\mathrm{B}(\alpha, \beta)}
\end{equation}
where $x$ is a probability and $\mathrm{B}(\alpha, \beta)$ is the beta
function
\begin{equation}
\mathrm{B}(\alpha, \beta) = \int_0^1 p^{\alpha - 1}(1-p)^{\beta - 1}
dp
\end{equation}
One may see that multiplying the density by the likelihood
\begin{equation}
x^m (1-x)^{n-m}
\end{equation}
of a particular sequence of $n$ experiments with $m$ positive
outcomes, is also a Beta distribution
% $$
% f(x; m+\alpha, n-m+\beta) = \frac{x^{m+\alpha - 1} (1-x)^{n-m+\beta - 1}}
%                                  {\mathrm{B}(m+\alpha, n-m+\beta)}
% $$
\begin{equation}
f(x; m+\alpha, n-m+\beta) \propto x^{m+\alpha - 1} (1-x)^{n-m+\beta - 1}
\end{equation}

\subsubsection{Cumulative Distribution Function}
The cumulative distribution function (cdf) of the Beta distribution is

\begin{equation}
I_x(\alpha, \beta) = \frac{\mathrm{B}(x; \alpha,
  \beta)}{\mathrm{B}(\alpha, \beta)}
\end{equation}
where $\mathrm{B}(x; \alpha, \beta)$ is the incomplete beta function
\begin{equation}
\mathrm{B}(x; \alpha, \beta) = \int_0^x p^{\alpha - 1}(1-p)^{\beta -
  1} dp
\end{equation}
$I_x$ is also called the regularized incomplete beta function.

\section{Partial Operator Induction with Beta Distributions }
\label{part-op}

In this section we introduce a specialization of Solomonoff Operator
Induction for partial operators describing second order distributions.
% Some ideas can be borrowed from the Missing Data literature
% \cite{Schafer02missingdata}, in particular.
% Let us now introduced a specialization of Solomonoff Operator
% Induction.  There are three aspects of that specialization. First, the
% operators output second order probabilities. Second, they are Beta
% distributions. And third, they are partial, that is they are only
% known for a subset of the observations.

% These 3 aspects reflects to our OpenCog realities... blahblah

\subsection{Second Order Probability Estimate}

Let us first modify the Solomonoff Operator Induction probability
estimate to become a second order probability estimate. This is
crucial to maintain the uncertainty surrounding that estimate.
% For instance if the estimate is about the probability
% of achieving some goal given some action in some context, the second
% order can be utilize to properly balance exploration and exploitation
% via Thompson sampling \cite{Leike17OnThompson}.
It directly follows from Eq. \ref{sol-eas} of Section
\ref{sol-op-ind}, that the cumulative distribution function of the
probability estimate of observing answer $A_{n+1}$ given question
$Q_{n+1}$ is
\begin{equation}
  \label{sol-cdf}
\hat{cdf}(A_{n+1}|Q_{n+1})(x) = \sum_{O^j(A_{n+1}|Q_{n+1}) \le x} a_0^j l^j
\end{equation}
Due to $O^j$ not being complete in general
$\hat{cdf}(A_{n+1}|Q_{n+1})(1)$ may not be equal to 1. It means that
some normalization will need to take place in practice. That is even
more true in our case since, as will be shown further below, the
operators taken into consideration are restricted to a subclass.
Also, obviously the continuity or the differentiability of
$\hat{cdf}(A_{n+1}|Q_{n+1})$ do not generally hold. What matters is
that a spread of probabilities is represented to properly account for
the uncertainty of that estimate. It is expected that the breadth
would be wide at first, and progressively shrinks, fluctuating
depending on the novelty of the contexts, as measure as more questions
and answers get collected.

\subsection{Continuous Parameterized Operators}

% Although Solomonoff Operator Induction supposedly considers partial
% operators (these are enumerable, while the set of complete operators
% is not), any partial operator that happens not to compute the
% probability of $A_i$ for some given $Q_i$ it will simply be entire
% dismissed in equation [REF]. In practice however partial models can be
% useful. Some partial models, only true in restricted contexts can
% nevertheless be good predictors and we need a way to take them into
% account. We will provide some suggestions to address that.

% Solomonoff Universal Operator Induction addresses well the problem of
% combining operators when these are complete, that is defined for each
% question. In our case however, since our operators are mere second
% order conditional probabilities, the distribution (or rather second
% order distributions) over answers are only known for some questions.

% To reuse our example above, the first rule tells us nothing about the
% probability of the temperature rising when the sun is not shining. So
% do the other rules when their conditions are not satisfied. For that
% reason we cannot apply Universal Operator Induction as is.

% To address that we suggest to artificially complete these operators by
% either

% 1. an uninformative prior
% 2. or a fictive perfect predictor

% If we were to see operators are programs, then the two options would
% be seen as

% 1. uninformative prior

% if condition
% then pdf
% else Jeffreys

% 2. fictive perfect predictor

% if condition
% then pdf
% else 1

% Let's first treat the fictive perfect predictor case.

% Let's reformulate Solomonoff Universal Operator Induction considering
% operators calculating second probabilities instead of probabilities.
% Let $cdf^j(A_i|Q_i)$ be the cumulative distribution function
% representing the second order conditional probability of $A_i$ knowing
% $Q_i$ according to the $j^{th}$ operator.

% \begin{equation}
% P(A_{i+1}|Q_{i+1}) = \sum_j P(p_j)
% \end{equation}

%%%%%%%%%%%%%%%% Let redo that %%%%%%%%%%%%%%

Let us now extend this for parameterized operators, so that each
operator is a second order distribution. Let us consider a subclass of
parameterized operators such that, if $p$ is the parameter of operator
$O^j_p$, the result of the conditional probability of $A_{n+1}$ given
$Q_{n+1}$ is
\begin{equation}
  \label{op}
O^j_p(A_{n+1}|Q_{n+1})=p
\end{equation}
% and the result of $O^j_p$ on another pair of question answer is
% \begin{equation}
%   \label{op}
% O^j_p(A_i|Q_i)=p
% \end{equation}
% TODO is $A_i=A_{n+1}$, $1-p$ otherwise.
We do that to later consider Beta distribution operators. The reason
for this assumption will become clearer in Section \ref{beta-op}.
% of this assumption may seem unclear, but that is essentially how
% truth values are defined in OpenCog. This will hopefully become
% clearer later on.
% Similar in spirit to integrating out continuous parameters [REF], with
% the exception that we build a cdf as opposed to an expectation.
Given that assumption, the cumulative distribution function of the
estimate $\hat{cdf}(A_{n+1}|Q_{n+1})$ becomes
\begin{equation}
  \label{consol}
  \hat{cdf}(A_{n+1}|Q_{n+1})(x) = \sum_j a_0^j \int_0^x f_p l_p^j dp
\end{equation}
where $f_p$ is the prior density of $p$, and
$l_p^j= \prod_{i=1}^{n} O^j_p(A_i|Q_i)$ is the likelihood of the data
according to the $j^{th}$ operator with parameter $p$.
\begin{proof}
  Consider continuous families of parameterized operators combined
  with Eq. \ref{op}. Let us start with the discrete case
\begin{equation}
  \hat{cdf}(A_{n+1}|Q_{n+1})(x) = \sum_{O^j_p(A_{n+1}|Q_{n+1})\le x}
  a_0^j f_p l_p^j \Delta p
\end{equation}
where the sum runs over all $j$ and $p$ by steps of $\Delta p$ such
that $O^j_p(A_{n+1}|Q_{n+1})\le x$. Assuming that $a_0^j$ does not
depends on $p$, it can be moved in its own sum
\begin{equation}
  \hat{cdf}(A_{n+1}|Q_{n+1})(x) = \sum_j a_0^j
  \sum_{O^j_p(A_{n+1}|Q_{n+1})\le x} f_p l_p^j \Delta p
\end{equation}
now the second sum only runs over $p$. Due to Eq. \ref{op} this can be
simplified into
\begin{equation}
  \hat{cdf}(A_{n+1}|Q_{n+1})(x) = \sum_j a_0^j \sum_{p\le x} f_p l_p^j
  \Delta p
\end{equation}
which is turns into Eq. \ref{consol} when $\Delta p$ tends to 0.
\end{proof}
% Here $f_p$ here is independent of the operator. It doesn't have to be
% necessarily the case, but for our purpose (which is to utilize some
% form of universal operator induction in OpenCog) this is enough. This
% will soon become clear.
% Using Solomonoff's notations, let's first define the cumulative
% distribution function of $A_{n+1}$. It is what we ultimately want as
% this will allow us to account for the uncertainty of the prediction,
% crucial for properly balancing exploration and exploitation when this
% knowledge is put in use [REF, REF]
Using continuous integration may seem like a departure from Solomonoff
Induction. First, it does not correspond to a countable class of
models. And second, the Kolmogorov complexity of $p$, that would in
principle determine its prior, is likely chaotic and very different
than how priors are typically defined over continuous parameters in
Bayesian inference. In practice however integration is discretized and
values are truncated up to some fixed precision. Moreover any prior
can probably be made compatible with Solomonoff induction by selecting
an adequate Turing machine of reference.
\subsection{Operators as Beta Distributions}
\label{beta-op}
We have now all we need to model our rules, second order conditional
probabilities, as operators.

First, we need to assume that operators are partial, that is the
$j^{th}$ operator is only defined for a subset of $n^j$ questions,
those that meet the conditions of the rule. For instance, with the
rule
\begin{itemize}
\item If the sun shines, then the temperature rises
\end{itemize}
questions and answers pertaining to what happens at night will be
ignored.

Second, we assume that answers are Boolean, so that $A_i\in \{0, 1\}$
for $i \in [1, n+1]$. In reality, OpenCog rules manipulate predicates
(generally fuzzy predicates but that can be let aside), and the
questions they represent are: if some instance holds property $R$,
what are the odds that it holds property $S$. We simplify this by
fixing $S$ so that the problem is reduced to finding $R$ that best
predict $S$, if $A_{n+1}=1$, or $\neg S$ if $A_{n+1}=0$. So the class
of operators under consideration are programs of the form
\begin{equation}
O^j_p(A_i|Q_i) = \text{if}\ R^j(Q_i)\ \text{then}\
\begin{cases}
  p, & \text{if}\ A_i = A_{n+1}\\
  1-p, & \text{otherwise}
\end{cases}
\end{equation}
where $R^j$ is the condition of the rule. This allows an operator to
be modeled as a Beta distribution, with cumulative distribution
function
\begin{equation}
  \label{O-cdf}
  cdf_{O^j} = I_x(m^j + \alpha, n^j-m^j+\beta)
\end{equation}
where $m^j$ is 
% Given that let us reduce the definition of our operators as follows
% \begin{equation}O^j_p(A_i=A_{n+1}|Q_i)=p\end{equation}
% \begin{equation}O^j_p(A_i\neq A_{n+1}|Q_i)=1-p\end{equation}
% if $O^j_p$ is defined for question $Q_i$, otherwise $Q^j(A_1|Q_i)$ is
% unknown. Let $m_j$ be
the number of times $A_i = A_{n+1}$ for the subset of $n^j$
questions such that $R^j(Q_i)$ is true. The parameters $\alpha$ and
$\beta$ are the parameters of the prior of $p$, itself a Beta
distribution. This corresponds in fact to the definition of OpenCog
Truth Values (see Chapter 4 of the PLN book \cite{Goertzel09PLN}).

\subsection{Handling Partial Operators}
When attempting to use such operators we still need to account for
their partiality. Although Solomonoff Operator Induction does in
principle encompass partial operators\footnote{more by necessity,
  since the set of partial operators are countable, while the set of
  complete ones are not}, it does so insufficiently, in our case
anyway. Indeed, if a given operator cannot compute the conditional
probability of some answer question pair, the contribution of that
operator may simply be ignored in the estimate. This does not work for
us since partial operators (rules over restricted contexts) might
carry significant predictive power and should not go to waste.

To the best of our knowledge, the existing literature does not cover
that problem. The Bayesian inference literature contains in-depth
treatments about how to properly consider missing data
\cite{Schafer02missingdata}. Unfortunately, they do not directly apply
here because our assumptions are different. In particular, here, data
omission depends on the model. However, the general principle of
modeling missing data and taking into account these models in the
inference process, can be applied. Let us attempt to do that in a by
explicitly representing the portion of the likelihood over the missing
data according to the $j^{th}$ operator by a term. In the rest of the
paper rather than calling these data \emph{missing} we prefer to
denominate them as \emph{unexplained} or \emph{unaccounted}, which
better captures our assumption. Let us also define a \emph{completion}
of $O^j_p$ as any program that can explain the unaccounted data.
\begin{definition}
  A \emph{completion} $C$ of $O^j_p$ is a program that completes
  $O^j_p$ for the unaccounted data, that is when $R^j(Q_i)$ is false
  $$
  \begin{array}{@{}ll@{}}
    O^j_{p, C}(A_i|Q_i) = &
                            \text{if}\ R^j(Q_i)\ \text{then}\
                            \begin{cases}
                              p, & \text{if}\ A_i = A_{n+1} \\
                              1-p, & \text{otherwise}
                            \end{cases}\\
                          & \text{else}\ C(A_i|Q_i)
  \end{array}
  $$
\end{definition}
Let us replace the likelihood in Eq. \ref{consol} by
\begin{equation}
  \label{new-lik}
l^j_p = p^{m^j}(1-p)^{n^j-m^j} r^j
\end{equation}
where the binomial term account for the likelihood of the explained
observations by the $j^{th}$ operator with parameter $p$, and $r^j$ is
a term that accounts for the likelihood of the unexplained
observations
\begin{equation}
  % r^j = \prod_{i \in \{i|\neg R^j(Q_i)\}} C^j(A_i|Q_i)
  r^j = \prod_{i \leq n\ \land\ \neg R^j(Q_i)} C^j(A_i|Q_i)
\end{equation}
assuming $C^j$ is the underlying completion of $O^j_p$ explaining the
unaccounted data. One may notice that $r^j$ does not depends on
$p$. Such assumption tremendously simplifies the analysis and is
somewhat reasonable to make. It means that the completion of the model
is independent on its pre-existing part.
% as the missing part of the model accounting for
% the missing data could simply be assume that We'll see later how to
% deal this term. Obviously $p^{m^j}$ and $p^{n^j-m^j}$ represent $p$ to
% the powers of $m^j$ and $n^j-m^j$ respectively, not superscripts.
Using Eq. \ref{new-lik} the cumulative distribution function of the
estimate of $A_{n+1}$ knowing $Q_{n+1}$ becomes
\begin{equation}
  \label{betaconsol}
  \hat{cdf}(A_{n+1}|Q_{n+1})(x) = \sum_j a_0^j \int_0^x f_p p^{m^j}(1-p)^{n^j-m^j}
  r^j dp
\end{equation}
Choosing a Beta distribution as the prior of $f_p$ simplifies the
equation as the posterior remains a Beta distribution
\begin{equation}
f_p = f(p; \alpha, \beta)
\end{equation}
where $f$ is the pdf of the Beta distribution as define in
Eq. \ref{beta-pdf}. Usual priors are Bayes' with $\alpha = 1$ and
$\beta = 1$, Haldane's with $\alpha = 0$ and $\beta = 0$ and Jeffreys'
with $\alpha = \frac{1}{2}$ and $\beta = \frac{1}{2}$. The latter is
the most accepted due to being \emph{uninformative} in some sense
\cite{Jeffreys46Invariant}. We do not need to commit to a particular
one at that point and let the parameters $\alpha$ and $\beta$ free,
giving us
\begin{equation}
  \label{betaconsol-1}
  \hat{cdf}(A_{n+1}|Q_{n+1})(x) = \sum_j a_0^j \int_0^x
  \frac{p^{\alpha - 1}(1-p)^{\beta - 1}}{\mathrm{B}(\alpha, \beta)}
  p^{m^j}(1-p)^{n^j-m^j} r^j dp
\end{equation}
% Let us assume that $r^j_p$ is independent of $p$, thus $r^j=r^j_p$ for
% any $p$. This may seem like an unrealistic assumption, and it is, but
% we allow ourselves to make it for sake of simplicity. This will be
$r^j$ can be moved out of the integral and the constant
$\mathrm{B}(\alpha, \beta)$ can be ignored on the ground that our
estimate will require normalization anyway
% do not need to worry about multiplicative
% constants because the prior over the operators is a semi-measure
% anyway, and in practice we will need to normalize the equation so that
% $cdf_{A_{n+1}}(1) = 1$. Given these reasons Eq. \ref{betaconsol-1} can
% be simplified into
\begin{equation}
  \label{betaconsol-2}
  \hat{cdf}(A_{n+1}|Q_{n+1})(x) \propto \sum_j a_0^j r^j
  \int_0^x p^{m^j+\alpha - 1}(1-p)^{n^j-m^j+\beta - 1} dp
\end{equation}
$\displaystyle\int_0^x p^{m^j+\alpha - 1}(1-p)^{n^j-m^j+\beta - 1} dp$
is the incomplete Beta function with parameters $m^j+\alpha$ and
$n^j-m^j+\beta$, thus
\begin{equation}
  \label{betaconsol-3}
  \hat{cdf}(A_{n+1}|Q_{n+1})(x) \propto \sum_j a_0^j r^j
  B(x; m^j+\alpha, n^j-m^j+\beta)
\end{equation}
Using the regularized incomplete beta function we obtain
\begin{equation}
  \label{betaconsol-4}
  \hat{cdf}(A_{n+1}|Q_{n+1})(x) \propto \sum_j a_0^j r^j
  I_x(m^j+\alpha, n^j-m^j+\beta)
  B(m^j+\alpha, n^j-m^j+\beta)
\end{equation}
As $I_x$ is the cumulative distribution function of $O^j$
(Eq. \ref{O-cdf}), we finally get
\begin{equation}
  \label{betaconsol-5}
  \hat{cdf}(A_{n+1}|Q_{n+1})(x) \propto \sum_j a_0^j r^j cdf_{O^j}(x)
  B(m^j+\alpha, n^j-m^j+\beta)
\end{equation}

We have expressed our cumulative distribution function estimate as an
averaging of the cumulative distribution functions of the
operators. This averaging is hopefully close to optimal (since the
operators are a subclass of Turing-complete operators optimality
cannot be guarantied), and most importantly it captures the
uncertainty of the estimate.\\

We still need to address $r^j$, the likelihood of the unaccounted
data. In theory, the right way to model $r^j$ would be to consider all
possible completions of the $j^{th}$ operator, but that is
intractable. One would be tempted to simply ignore $r^j$, however, as
we have already observed in some preliminary experiments, this gives
an unfair advantage to rules that have a lot of unexplained data, and
thus make them more prone to overfitting. This is true even in spite
of the fact that such rules naturally exhibit more uncertainty due to
having less evidence.

\subsection{Perfectly Explaining Unaccounted Data}
Instead we attempt to consider the most prominent completions.  For
now we consider completions that perfectly explain the unaccounted
data. Moreover, to simplify further, we assume that unaccounted
answers are entirely determined by their corresponding questions. This
is generally not true, the same question may relate to different
answers. But under such assumptions $r^j$ becomes 1. This may seem
equivalent to ignoring $r^j$ unless the complexity of the completion
is taken into account. What that means is that we must consider, not
only the complexity of the rule but also the complexity of the
completion. Unfortunately calculating that complexity (that is the
Kolmogorov complexity) of is intractable. To work around that we
estimate it with a simple heuristic
\begin{equation}
a^j_0 = K(O^j) + v_j^{(1-k)}
\end{equation}
where $K(O^j)$ is the Kolmogorov complexity of the $j^{th}$ operator,
$v_j$ is the size of the unaccounted data by the $j^{th}$ operator,
and $k$ is a \emph{compressability} parameter. If $k=0$ then the
unaccounted data are incompressible. If $k=1$ then the unaccounted
data can be compressed to a single bit. It is a very crude heuristic
and is not parameter free, but it is simple and computationally
lightweight. When applied to experiments (not described here due to
their embryonic nature and due to space limitation) a value of $k=0.5$
was actually shown to be satisfactory.

\section{Conclusion}
\label{con}
We have introduced a specialization of Solomonoff Operator Induction
over operators with the particularities of being partial and being
modeled by Beta distributions. While doing so we have uncovered an
interesting problem, how to include the contributions of partial
operators in the averaging. This problem appears to have no obvious
solution, is manifestly under-addressed by the research community, and
is yet important in practice. Although the solution we provide is very
lacking (crudely estimating the Kolmogorov complexity of a perfect
completion) we hope that it may motivate further research.
% Also, our estimate has already been used in the context of inference
% control meta-learning within the OpenCog framework and, in spite of
% being very early stage, has shown positive results.
% Also, we have started experimenting with our estimate in the context
% inference control meta-learning within OpenCog, and the results,
% albeit very early stage, seem promising. For instance in our early
% experiments a value of $k=0.5$ was shown to be satisfactory.
% The proposed methodology has already been used in the context of
% inference control meta-learning within the OpenCog framework and, in
% spite of being very early stage, has shown positive results.
% Subsequent publications will be make available as that work
% progresses.
Even though, ultimately, it is expected that this problem is hard
enough that it may require some form of meta-learning
\cite{Goertzel16Probabilistic}, improvements in the heuristic by, for
instance, considering completions reusing available models that do
explain some unaccounted data could help.
% , as well as enrich the operator expressiveness by for instance
% replacing Beta distributions by Dirichlet distributions.

Experiments using this estimate are currently being carried out in the
context of inference control meta-learning within the OpenCog
framework and will be presented in future publications.

%
% ---- Bibliography ----
%
\bibliographystyle{splncs04}
\bibliography{my}

\end{document}
