%%%%%%%%%%%%%%%%%%%%%%%%%%%%%%%%%%%%%%%%%%%%%%%%%%%%%%%%%%%%%%%%%%%%%%%
%% AGI-22 paper about temporal and procedural reasoning with OpenCog %%
%%%%%%%%%%%%%%%%%%%%%%%%%%%%%%%%%%%%%%%%%%%%%%%%%%%%%%%%%%%%%%%%%%%%%%%

\documentclass[runningheads]{llncs}
%
\usepackage{graphicx}
\usepackage{amsmath}
\usepackage{bussproofs}
\usepackage{cite}
% Used for displaying a sample figure. If possible, figure files should
% be included in EPS format.
%
% If you use the hyperref package, please uncomment the following line
% to display URLs in blue roman font according to Springer's eBook style:
% \renewcommand\UrlFont{\color{blue}\rmfamily}

% Commands for Atomese code
\newcommand{\SP}{\;\;\;}
\newcommand{\TTrue}{\textit True}
\newcommand{\TFalse}{\textit False}
\newcommand{\TAtom}{\textit Atom}
\newcommand{\TEval}{\textit Evaluation}
\newcommand{\TLamb}{\textit Lambda}
\newcommand{\TAnd}{{\textit And}}
\newcommand{\TNot}{{\textit Not}}
\newcommand{\TImpl}{{\textit Implication}}
\newcommand{\TPredImpl}{{\textit PredictiveImplication}}
\newcommand{\TSeqAnd}{{\textit SequentialAnd}}
\newcommand{\TTV}{\textit TV}
\newcommand{\TTVPi}{\textit TV_i^P}
\newcommand{\TTVQi}{\textit TV_i^Q}
\newcommand{\TBTV}{\langle \TTV \rangle}
\newcommand{\TBTVPi}{\langle \TTVPi \rangle}
\newcommand{\TBTVQi}{\langle \TTVQi \rangle}
\newcommand{\Tstrength}{\textit s}
\newcommand{\Tconf}{\textit c}

\begin{document}
%
\title{Temporal and Procedural Reasoning for Rational Agent Control in
  OpenCog}

%\titlerunning{Abbreviated paper title}
% If the paper title is too long for the running head, you can set
% an abbreviated paper title here
%
\author{Nil Geisweiller
  %\orcidID{0000-0001-5041-6299}
  \and Hedra Yusuf}
%
\authorrunning{N. Geisweiller et al.}
% First names are abbreviated in the running head.
% If there are more than two authors, 'et al.' is used.
%
\institute{ SingularityNET Foundation, The
  Netherlands\\ \email{\{nil,hedra\}@singularitynet.io}}
%
\maketitle              % typeset the header of the contribution
%

\begin{abstract}
  TODO

  \keywords{Temporal \and Procedural \and Reasoning \and Probabilistic
    Logic Networks \and OpenCog}
\end{abstract}

\section{Introduction}

The goal of this project is to make an agent as rational as possible,
not necessarily as efficient as possible.  This stems from the concern
that in order to autonomously gain efficiency the agent must first be
able to make the best possible decisions, starting first in the outer
world, and then in the inner world.

The paper presents

The agent starts in a completely unknown environment

The idea is that reasoning is used at all levels, discovering patterns
from raw observations, building plans and making decisions.

\section{Contributions}

The contributions of that paper are:
\begin{enumerate}
\item Build upon existing temporal reasoning framework defined in
  Chap.14 [TODO: cite PLN book].
\item Design an architecture for controlling an agent based on that
  temporal reasoning extension.
\end{enumerate}

\section{Outline}

\begin{enumerate}
\item Temporal reasoning
\item ROCCA
\item Minecraft experiment
\end{enumerate}

\section{Recall: Probabilistic Logic Networks}

PLN, which stands for Probabilistic Logic Networks, is a mixture of
predicate and term logic that has been probabilitized to properly
handle uncertainty.  It has two types of rules
\begin{enumerate}
\item one type for introducing relationships from direct observations,
\item the other for introducing relationships from existing
  relationships.
\end{enumerate}
As such it is especially suited for building an ongoing understanding
of an unknown environment (using direct introduction rules), and then
planning in that environment (using indirect introduction rules).

\subsection{Elementary Notions}

%% Let us first recall the minimum portion of PLN we will need to
%% describe the temporal logic used in this paper.

Graphically speaking, PLN statements are
sub-hypergraphs\footnote{because links can point to links, not just
nodes} made of links and nodes, called \emph{Atoms}, decorated with
\emph{Truth Values} that can be understood as probabilities
incorporating uncertainties
%% ,
%% that can be understood as probabilities incorporating uncertainties,
%% more specifically second order probability distributions\footnote{it
%% should be noted that even though these predicates seems crisp, they
%% are actually probabilistic because probabilities can be assigned to
%% their evaluations, but that is not our concern for this paper}.
\cite{TODO}.  Syntactically speaking however PLN statements are not
very different from statements expressed in another logic, except that
they are usually formatted in prefixed-operator indented-argument
style to emphasize their graphical nature and leave room for truth
values.  There is a large variety of constructs for PLN, here we will
focus primarily on constructs for manipulating predicates.  Let us
recall that predicates are functions that take tuples of Atoms and
output boolean values
$$P, Q, R, \hdots: \TAtom^n \mapsto \{\TTrue, \TFalse\}$$ Within
predicate constructs there are two classes of operators
\begin{enumerate}
\item one for defining predicate from instances, such as $\TEval$ and
  $\TLamb$,
\item and another for combining existing predicates, such as $\TAnd$,
  $\TNot$ and $\TImpl$.
\end{enumerate}
Let us present these operators below, corresponding to the minimum
subset we will need in the rest of the paper.
\begin{itemize}
\item Evaluation:
  $$
  \begin{array}{l}
    \TEval\ \TBTV\\
    \SP P\\
    \SP e\\
  \end{array}
  $$
  states that $P(e)$ outputs $\TTrue$ to a degree set by the truth value
  $\TTV$.
\item Lambda:
  $$
  \begin{array}{l}
    \TLamb\ \TBTV\\
    \SP x\\
    \SP P(x)\\
  \end{array}
  $$
  is a predicate constructor with variable $x$ and predicate body
  $P(x)$, where the true value $\TTV$ corresponds to the probability
  $\mathbf{Pr}(P)$ of $P(x)$ to output $\TTrue$ for a random input.
\item Conjunction:
  $$
  \begin{array}{l}
    \TAnd\ \TBTV\\
    \SP P\\
    \SP Q\\
  \end{array}
  $$
  represents the predicate obtained by taking the conjunction of
  $P$ and $Q$, or equivalently the indicator function corresponding to
  the intersection of the \emph{satisfying sets} of $P$ and $Q$.
  The truth value $\TTV$ then represents an estimate of the
  probability $\mathbf{Pr}(P,Q)$ of the conjunction of $P$ and $Q$.
\item Negation:
  $$
  \begin{array}{l}
    \TNot\ \TBTV\\
    \SP P\\
  \end{array}
  $$
  represents the negation of $P$, or equivalently the indicator
  function corresponding to the complement of the satisfying set of
  $P$. The truth value $\TTV$ then represents an estimate of
  the probability $\mathbf{Pr}(\neg P)$ of the negation of $P$.
\item Implication:
  $$
  \begin{array}{l}
    \TImpl\ \TBTV\\
    \SP P\\
    \SP Q\\
  \end{array}
  $$
  represents the predicate $Q$ conditioned on $P$, that is only
  defined for instances $x$ for which $P(x)$ is $\TTrue$.  The truth
  value $\TTV$ then represents an estimate of the conditional
  probability $\mathbf{Pr}(Q|P)$.  There is some subtleties to take
  into account due to the fact $P(x)$ can actually be partially true
  (stated by the truth value of $\TEval$ as explained above), but this
  all resolves nicely by assuming degrees of truth are probabilistic.
  More is explained about that below.
\end{itemize}
Truth values are fundamentally second order probability
distributions. However in practice they are usually represented by two
numbers, a strength and a confidence, both ranging from 0 to 1.  The
strength represents a probability while the confidence represents a
precision over that probability.  Underneath, strength and confidence
can be mapped into a second order distribution such as a Beta
distribution [TODO: add figure].

\subsection{Inference Rules}

Beside operators, inferences rules are used to construct PLN
statements and calculate their truth values.  They mainly fall into
two categories, direct and indirect.  Direct rules infer abstract
knowledge from direct evidence, while indirect rules infer knowledge
by combining existing abstractions, themselves inferred directly or
indirectly.  There are dozens of inference rules but for now we will
only recall two which are needed for the paper:
\begin{enumerate}
\item \emph{Implication Direct Introduction Rule}
\item \emph{Deduction Rule}
\end{enumerate}

\subsubsection{The Implication Direct Introduction Rule (IDI)} takes $\TEval$
links as premises and produces an $\TImpl$ link as conclusion,
formally depicted as the proof tree
{\small
\begin{prooftree}
  \AxiomC{$
    \begin{array}{l}
      \TEval\ \TBTVPi\\
      \SP P\\
      \SP E_i\\
    \end{array}
    $}
  \AxiomC{$\hdots$}
  \AxiomC{$
    \begin{array}{l}
      \TEval\ \TBTVQi\\
      \SP Q\\
      \SP E_i\\
    \end{array}
    $}
  \RightLabel{(IDI)}
  \TrinaryInfC{$
    \begin{array}{l}
      \TImpl\ \TBTV\\
      \SP P\\
      \SP Q\\
    \end{array}
    $}
\end{prooftree}}
Assuming perfectly reliable direct evidence\footnote{dealing with
unreliable direct evidence involves expensive convolution products and
is outside of the scope of this paper.} then the resulting truth value
is calculated as follows
$$\TTV.\Tstrength = \frac{\sum_{i=1}^n f_\wedge(\TTVPi.\Tstrength, \TTVQi.\Tstrength)}{\sum_{i=1}^n \TTVPi.\Tstrength}$$
$$\TTV.\Tconf = \frac{n}{n+k}$$ where $\TTV.\Tstrength$ and
$\TTV.\Tconf$ represents the strength and the confidence of $\TTV$
respectively, $k$ is a system parameter, and $f_\wedge$ is a function
embodying a probabilistic assumption about the intersection of the
events corresponding to the \emph{probabilitized degrees of truth} of
$P(E_i)$ and $Q(E_i)$.  Such function typically ranges from the product
function (perfect independence) to the $\min$ function (perfect
overlap).

\subsubsection{The Deduction Rule}

NEXT

\section{Temporal Logic}

The temporal logic used here builds upon what is described in the
Chapter 14 of the PLN book [TODO: cite].  Let us define that

\subsection{Mapping Temporal into Atemporal Statements}



\section{Rational OpenCog Controlled Agent (ROCCA)}

\section{Experiment with Simple Minecraft Environment}

%% Does not do justice and is not meant to demonstrate the prowess of
%% ROCCA

\section{Conclusion}

%
% ---- Bibliography ----
%
\bibliographystyle{splncs04} \bibliography{TPRC}

\end{document}
