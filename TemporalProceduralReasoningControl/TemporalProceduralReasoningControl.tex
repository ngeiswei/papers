%%%%%%%%%%%%%%%%%%%%%%%%%%%%%%%%%%%%%%%%%%%%%%%%%%%%%%%%%%%%%%%%%%%%%%%
%% AGI-22 paper about temporal and procedural reasoning with OpenCog %%
%%%%%%%%%%%%%%%%%%%%%%%%%%%%%%%%%%%%%%%%%%%%%%%%%%%%%%%%%%%%%%%%%%%%%%%

\documentclass[runningheads]{llncs}
%
\usepackage{graphicx}
\usepackage{amsmath}
\usepackage{bussproofs}
\usepackage{cite}
% Used for displaying a sample figure. If possible, figure files should
% be included in EPS format.
%
% If you use the hyperref package, please uncomment the following line
% to display URLs in blue roman font according to Springer's eBook style:
% \renewcommand\UrlFont{\color{blue}\rmfamily}

% Commands for Atomese code
\newcommand{\SP}{\;\;\;}
\newcommand{\TP}{\textit P}
\newcommand{\TQ}{\textit Q}
\newcommand{\TR}{\textit R}
\newcommand{\TTrue}{\textit True}
\newcommand{\TFalse}{\textit False}
\newcommand{\TAtom}{\textit Atom}
\newcommand{\TImpl}{{\textit Implication}}
\newcommand{\TAnd}{{\textit And}}
\newcommand{\TNot}{{\textit Not}}
\newcommand{\TPredImpl}{{\textit PredictiveImplication}}
\newcommand{\TSeqAnd}{{\textit SequentialAnd}}
\newcommand{\TTV}{\langle \textit TV \rangle}

\begin{document}
%
\title{Temporal and Procedural Reasoning for Rational Agent Control in
  OpenCog}

%\titlerunning{Abbreviated paper title}
% If the paper title is too long for the running head, you can set
% an abbreviated paper title here
%
\author{Nil Geisweiller
  %\orcidID{0000-0001-5041-6299}
  \and Hedra Yusuf}
%
\authorrunning{N. Geisweiller et al.}
% First names are abbreviated in the running head.
% If there are more than two authors, 'et al.' is used.
%
\institute{ SingularityNET Foundation, The
  Netherlands\\ \email{\{nil,hedra\}@singularitynet.io}}
%
\maketitle              % typeset the header of the contribution
%

\begin{abstract}
  TODO

  \keywords{Temporal \and Procedural \and Reasoning \and Probabilistic
    Logic Networks \and OpenCog}
\end{abstract}

\section{Introduction}

The goal of this project is to make an agent as rational as possible,
not necessarily as efficient as possible.  This stems from the concern
that in order to autonomously gain efficiency the agent must first be
able to make the best possible decisions, starting first in the outer
world, and then in the inner world.

The paper presents

The agent starts in a completely unknown environment

The idea is that reasoning is used at all levels, discovering patterns
from raw observations, building plans and making decisions.

\section{Contributions}

The contributions of that paper are:
\begin{enumerate}
\item Build upon existing temporal reasoning framework defined in
  Chap.14 [TODO: cite PLN book].
\item Design an architecture for controlling an agent based on that
  temporal reasoning extension.
\end{enumerate}

\section{Outline}

\begin{enumerate}
\item Temporal reasoning
\item ROCCA
\item Minecraft experiment
\end{enumerate}

\section{Probabilistic Logic Networks}

PLN, which stands for Probabilistic Logic Networks, is a mixture of
predicate and term logic that has been probabilitized to properly
handle uncertainty.  It has two types of rules
\begin{enumerate}
\item one type for introducing relationships from direct observations,
\item the other for introducing relationships from existing
  relationships.
\end{enumerate}
As such it is especially suited for building an ongoing understanding
of an unknown environment (using direct introduction rules), and then
planning in that environment (using indirect introduction rules).

\subsection{Recall}

Let us first recall the minimum portion of PLN we will need to
describe the temporal logic used in this paper.

Graphically speaking a PLN statement is a graph made of links and
nodes, called \emph{Atoms}, decorated with \emph{Truth Values}
%% ,
%% that can be understood as probabilities incorporating uncertainties,
%% more specifically second order probability distributions\footnote{it
%% should be noted that even though these predicates seems crisp, they
%% are actually probabilistic because probabilities can be assigned to
%% their evaluations, but that is not our concern for this paper}.
\cite{TODO}.  It is said that such graph is a hypergraph or a
metagraph because links can point to links \cite{TODO}.  Syntactically
speaking however, a PLN statement is not very different than a
statement expressed in another logic.  Here will focus primarily on
Atoms representing predicates
$$P, Q, R, \hdots: \TAtom^n \mapsto \{\TTrue, \TFalse\}$$ as well as a
small subset of connectors operating on these predicates, recalled
below.
\begin{itemize}
\item Conjunction:
  $$
  \begin{array}{l}
    \TAnd\ \TTV\\
    \SP \TP\\
    \SP \TQ\\
  \end{array}
  $$
  represents the predicate obtained by taking the conjunction of
  $P$ and $Q$, or equivalently the indicator function corresponding to
  the intersection of the \emph{satisfying sets} of $P$ and $Q$.  The
  truth value $\textit{TV}$ then represents an estimate of the
  probability $\mathbf{Pr}(P,Q)$ of the conjunction of $P$ and $Q$.
\item Negation:
  $$
  \begin{array}{l}
    \TNot\ \TTV\\
    \SP \TP\\
  \end{array}
  $$
  represents the negation of $P$, or equivalently the indicator
  function corresponding to the complement of the satisfying set of
  $P$. The truth value $\textit{TV}$ then represents an estimate of
  the probability $\mathbf{Pr}(\neg P)$ of the negation of $P$.
\item Implication:
  $$
  \begin{array}{l}
    \TImpl\ \TTV\\
    \SP \TP\\
    \SP \TQ\\
  \end{array}
  $$ represents the predicate $Q$ conditioned on $P$, that is only
  defined for the instances $x$ for which $P(x)$ is true.  The truth
  value $\textit{TV}$ then represents an estimate of the conditional
  probability $\mathbf{Pr}(Q|P)$.
\end{itemize}
Truth values are, fundamentally speaking, second order probability
distributions. However in practice they are usually represented by two
numbers, a strength and a confidence, both ranging from 0 to 1.  The
strength represents a probability while the confidence represents a
precision over that probability.  Underneath, strength and confidence
can be mapped into a second order distribution via the Beta
distribution [TODO: add figure].

\subsection{Inference Rules}

\subsubsection{Direct Introduction Rules}

NEXT

\subsubsection{Indirect Introduction Rules}

\section{Temporal Logic}

The temporal logic used here builds upon what is described in the
Chapter 14 of the PLN book [TODO: cite].  Let us define that

\subsection{Mapping Temporal into Atemporal Statements}



\section{Rational OpenCog Controlled Agent (ROCCA)}

\section{Experiment with Simple Minecraft Environment}

\section{Conclusion}

%
% ---- Bibliography ----
%
\bibliographystyle{splncs04} \bibliography{TPRC}

\end{document}
